\documentclass[xcoler=dvipsnames, aspectratio=169]{beamer}
\usepackage{3191Style}
\date{Determinants}

\begin{document}
    \begin{frame}{Determinant Definition}
        \small
        \begin{defn}
            The \rText{Determinant} is a function given by
            \[
                \textnormal{det}:\R^{n\times n}\rightarrow\R
            \]
            with the following properties:
            \begin{enumerate}
                \pause\item Doing a row replacement does not change det$(A)$
                \pause\item Scaling a row of $A$ by a scaler $c$ multiplies the determinant by $c$.
                \pause\item Swapping two rows of a matrix multiplies the determinant by $-1$.
                \pause\item The determinant of $I_n$ is 1.
            \end{enumerate}
        \end{defn}
        \pause
        \begin{tcolorbox}
            In fact, the determinant the unique function with these properties!\pause\ 
            But, proving this with the tools we have is difficult, so we will just take this for granted.
        \end{tcolorbox}
    \end{frame}
    \begin{frame}{Method to Compute the Determinant}
        We can compute the determinant by doing:
        \begin{enumerate}
            \pause\item Reduce to reduced row echelon form
            \pause\item Do operations in reverse following previous rules!
        \end{enumerate}
    \end{frame}
    \begin{frame}{Practice!}
        Find the determinant of 
        \[
            A = \begin{bmatrix}
                1 & 2\\
                3 & 4
            \end{bmatrix}\qquad\onslide<9->\det{A} = -2\onslide<1->
        \]\pause
        \begin{columns}
            \column{.5\textwidth}
            \[
                \begin{bmatrix}
                    1 & 2\\
                    3 & 4
                \end{bmatrix}\pause\xrightarrow{R_2=R_2-3R_1}\begin{bmatrix}
                    1 & 2\\
                    0 & -2
                \end{bmatrix}
            \]
            \[
                \pause\xrightarrow{R_2=\frac{-R_2}{2}}\begin{bmatrix}
                    1 & 2\\
                    0 & 1
                \end{bmatrix}
            \]
            \[
                \pause\xrightarrow{R_1=R_1-2R_2}\begin{bmatrix}
                    1 & 0\\
                    0 & 1
                \end{bmatrix}
            \]
            \column{.5\textwidth}
            \onslide<8->
            $\det{A} = -2$
            \vspace{20pt}

            \onslide<7->
            $\det{A} = 1$
            \vspace{20pt}

            \onslide<6->
            $\det{A} = 1$

        \end{columns}
    \end{frame}
    \begin{frame}{Now you try!}
        Find the determinant of 
        \[
            A = \begin{bmatrix}
                1 & 2\\
                1 & 3
            \end{bmatrix}
        \]
        \vspace{130pt}
    \end{frame}
    \begin{frame}{Special Types of Matrices}
        To make our later discussion easier, we define two new kinds of matrices\pause
        \begin{columns}
            \column{.5\textwidth}
            Upper triangular
            \pause
            \[
                U = \begin{bmatrix}
                    u_{11} & u_{12} & \dots & u_{1,n-1}& u_{1n}\\
                    0 & u_{22} & \dots & u_{2,n-1} & u_{2n}\\
                    \vdots & \ddots & \ddots & \vdots & \vdots\\
                    0 & 0 & \ddots & u_{n-1.n-1} & u_{n-1,n}\\
                    0 & 0 & \dots & 0 & u_{nn}
                \end{bmatrix}
            \]
            \pause
            \begin{defn}
                A matrix $U\in\R^{n\times n}$ is \rText{Upper Triangular} if:\pause
                \[
                    u_{ij} = 0 \textnormal{ for all } 1\leq j < i \leq n
                \]
            \end{defn}
            \column{.5\textwidth}
            Lower triangular
            \pause
            \[
                L = \begin{bmatrix}
                    \ell_{11}   & 0 & \dots  &0           & 0 \\
                    \ell_{12}   & \ell_{22} & \ddots  &0           & 0 \\
                    \vdots      & \vdots    & \ddots & \ddots         & \vdots\\
                    \ell_{1,n-1}& \ell_{1n}         & \ddots & \ell_{n-1.n-1} & 0\\
                    \ell_{2,n-1}& \ell_{2n}         & \dots  & \ell_{n-1,n}   & \ell_{nn}
                \end{bmatrix}
            \]
            \pause
            \begin{defn}
                A matrix $L\in\R^{n\times n}$ is \rText{Lower Triangular} if:\pause
                \[
                    \ell_{ij} = 0 \textnormal{ for all } 1\leq i < j \leq n
                \]
            \end{defn}
        \end{columns}
    \end{frame}
    \begin{frame}{Determinant of a Matrix with a 0 row}
        A matrix with a $0$ row will look something like below
        \[
            A = \begin{bmatrix}
                a_{11} & a_{12} & a_{13} \\
                0 & 0 & 0\\
                a_{31} & a_{32} & a_{33}
            \end{bmatrix}
        \]\pause
        Recall that scaling a row of $A$ by a scaler $c$ multiplies the determinant by $c$.
        \pause

        \[
            A = \begin{bmatrix}
                a_{11} & a_{12} & a_{13} \\
                0 & 0 & 0\\
                a_{31} & a_{32} & a_{33}
            \end{bmatrix}\xrightarrow{R_2 = -R_2}\aMat{ccc}{
                a_{11} & a_{12} & a_{13} \\
                0 & 0 & 0\\
                a_{31} & a_{32} & a_{33}
            } = A
        \]
        \pause
        So,
        \[
            \det{A} = -\det{A}\pause\rightarrow 2\det{A} = 0\pause\rightarrow \det{A} = 0.
        \]
    \end{frame}
    \begin{frame}{Determinant of a Triangular Matrix}
        We will work with an upper triangular matrix, some good practice would be to 
        do these arguments but for lower triangular matrices!\pause

        Remember that we have two cases to think about for a square matrix, $A\in\R^{n\times n}$
        \begin{enumerate}
            \pause\item We have less than $n$ pivots
            \pause\item We have $n$ pivots
        \end{enumerate}\pause
        We claim that in both cases, the determinant of $A$ is the product of the elements
        on the diagonal
    \end{frame}
    \begin{frame}{Less than $n$ pivots}
        What does this look like?\pause
        \[
            A = \begin{bmatrix}
                a_{11} & a_{12} & a_{13}\\
                0 & 0 & a_{23}\\
                0 & 0 & a_{33}
            \end{bmatrix}\xrightarrow{R_2=R_2 - \frac{a_{23}}{a_{33}}R_3}\aMat{ccc}{
                a_{11} & a_{12} & a_{13}\\
                0 & 0 & 0\\
                0 & 0 & a_{33}
            }
        \]
        \pause

        There's a 0 row, so $\det{A} = 0$.\pause

        This idea extends to larger matrices too! Try to think about what that proof 
        would look like!
    \end{frame}
    \begin{frame}{$n$ pivots}
        What does this look like?\pause
        \[
            A = \begin{bmatrix}
                a_{11}  & a_{12}& a_{13}\\
                0       & a_{22}& a_{23}\\
                0       & 0     & a_{33}
            \end{bmatrix}\xrightarrow{R_2=R_2 - \frac{a_{23}}{a_{33}}R_3}\aMat{ccc}{
                a_{11}  & a_{12}& a_{13}\\
                0       & a_{22}& 0\\
                0       & 0     & a_{33}
            }\pause\xrightarrow{R_1 = R_1 - \frac{a_{13}}{a_{33}}R_3}\aMat{ccc}{
                a_{11}  & a_{12}& 0\\
                0       & a_{22}& 0\\
                0       & 0     & a_{33}
            }
        \]
        \[
            \pause\xrightarrow{R_1 = R_1 - \frac{a_{12}}{a_{22}}R_2}\aMat{ccc}{
                a_{11}  & 0     & 0\\
                0       & a_{22}& 0\\
                0       & 0     & a_{33}
            }\pause\xrightarrow[R_3=\frac{R_3}{a_{33}}]
            {R_1 = \frac{R_1}{a_{11}}, R_2 = \frac{R_2}{a_{22}}}
            \aMat{ccc}{
                1       & 0     & 0\\
                0       & 1     & 0\\
                0       & 0     & 1
            }\pause = B
        \]
        And:
        \pause
        \[
            \det{A} = \det{B}\cdot a_{11} \cdot a_{22} \cdot a_{33} = a_{11}a_{22}a_{33}
        \]
    \end{frame}
    \begin{frame}{General $2\times 2$ formula}
        \begin{theorem}
        Let 
        \[
            A = \begin{bmatrix}
                a & b\\
                c & d
            \end{bmatrix}
        \]
        Then
        \[
            \det{A} = ad-bc
        \]
        \end{theorem}
        \vspace{130pt}
    \end{frame}
    \iftoggle{showSolutions}{
        \begin{frame}{$a=0$ case}
            \begin{proof}
                If $a=0$, we need to have a pivot in $A_{11}$
                \pause
                \[
                    A = \aMat{cc}{
                        0 & b\\
                        c & d
                    }\pause\xrightarrow{R_1\leftrightarrow R_2}\aMat{cc}{
                        c & d\\
                        0 & b
                    } = B
                \]
                See that $\det{A} = -\det{B} = -bc\pause = ad - bc$.
            \end{proof}
        \end{frame}
    }{}
    \iftoggle{showSolutions}{
        \begin{frame}{$a\neq0$ case}
            \begin{proof}
                If $a\neq0$, we just need to eliminate $c$!
                \pause
                \[
                    A = \aMat{cc}{
                        a & b\\
                        c & d
                    }\pause\xrightarrow{R_2 = R_2 - \frac{c}{a}R_1}\aMat{cc}{
                        a & b\\
                        0 & d-\frac{bc}{a}
                    } = B
                \]
                See that $\det{A} = \det{B} = a\left(d-\frac{bc}{a}\right)\pause = ad - bc$.
            \end{proof}
        \end{frame}
    }{}
    \begin{frame}{General $3\times 3$ formula}
        We could derive this formula, but it would be easier with Section 4.2, which we will not 
        be covering in class.\pause
        \begin{theorem}
        Let 
        \[
            A = \begin{bmatrix}
                a & b & c\\
                d & e & f\\
                g & h & i
            \end{bmatrix}
        \]
        Then
        \[
            \det{A} = a(ei-fh) - b(di-fg) + c(dh-eg)
        \]
        \end{theorem}
    \end{frame}
    \begin{frame}{More Practice} 
        Determine if the determinant of the following systems is $0$ or not.
        \[
            A = \begin{bmatrix}
                1 & 2 & 3\\
                2 & 4 & 4\\
                1 & 2 & 2
            \end{bmatrix}\qquad\qquad B = \begin{bmatrix}
                1 & 2\\
                3 & 1
            \end{bmatrix}\qquad\qquad C = \begin{bmatrix}
                1 & 2 & 3 & 1\\
                2 & 4 & 4 & 2\\
                1 & 2 & 2 & 3\\
                0 & 0 & 0 & 1
            \end{bmatrix}
        \]
        \vspace{130pt}
    \end{frame}
    \begin{frame}{Determinant Properties}
        These are provable with what we have now, but we will take them for granted for now.
        I would think about how to demonstrate the second property though!

        Let $A,B\in\R^{n\times n}$.
        \begin{enumerate}
            \pause\item $A$ is invertible if and only if $\det{A} \neq 0$.
            \pause\item If $\det{A}=0$, then $A$ has linearly dependent rows and columns
            \pause\item $\det{AB} = \det{A}\det{B}$
            \pause\item $\det{A^\top}=\det{A}$
        \end{enumerate}
    \end{frame}
\end{document}
