% on slide 7.
\documentclass[xcoler=dvipsnames, aspectratio=169]{beamer}

\usepackage{3191Style}
% Date gives the title of the lecture
\date{Diagonalizability}

\begin{document}
    \begin{frame}{Diagonal Matrix Arithmetic}
        To motivate this section, let's look at how diagonal matrices multiply
        Let $D_1=\bMat{x&0&0\\0&y&0\\0&0&z},D_2=\bMat{a&0&0\\0&b&0\\0&0&c}$, see that\pause
        \[
            D_1D_2 = \bMat{x&0&0\\0&y&0\\0&0&z}\bMat{a&0&0\\0&b&0\\0&0&c} \pause =
            \bMat{x\\0\\0}\bMat{a&0&0} + \bMat{0\\y\\0}\bMat{0&b&0} + \bMat{0\\0\\z}\bMat{0&0&z}\pause
        \]
        \[
            \bMat{
                xa & 0 & 0\\
                0 & 0 & 0\\
                0 & 0 & 0
            }+\bMat{
                0 & 0 & 0\\
                0 & yb & 0\\
                0 & 0 & 0
            }+\bMat{
                0 & 0 & 0\\
                0 & 0 & 0\\
                0 & 0 & zc
            }\pause = \bMat{
                xa & 0 & 0\\
                0 & yb & 0\\
                0 & 0 & zc
            }
        \]
    \end{frame}
    \begin{frame}{Powers of Diagonal Matrices}
        From the previous slide, we see that for any diagonal
        $D=\bMat{
            x&0&0\\
            0&y&0\\
            0&0&z
        }$, we have
        \[
            D^n = \bMat{
            x^n&0&0\\
            0&y^n&0\\
            0&0&z^n
        }
        \]
    \end{frame}
    \begin{frame}{Diagonalizability}
        \begin{defn}
            \rText{Diagonalizable}: We say that a matrix $A\in\R^{n\times n}$ is 
            \bText{diagonalizable} if it is similar to a diagonal matrix.\pause

            Or equivalently, $A$ is \bText{diagonalizable} if there exists an invertible 
            $C\in\R^{n\times n}$, and diagonal $D\in\R^{n\times n}$ such that
            \[
                A = CDC^{-1}
            \]
        \end{defn}
        \pause\
        An application is in computing matrix powers!
        \[
            A^n = \left(CDC^{-1}\right)^n\pause =\underbrace{CDC^{-1}CDC^{-1}\cdots 
            CDC^{-1}CDC^{-1}}_{n\textnormal{ times}}\pause =CD^nC^{-1}
        \]
    \end{frame}
    \begin{frame}{Diagonalization Theorem}
        \small
        \begin{theorem}
            Let $A\in\R^{n\times n}$. Then, $A$ is diagonalizable if and only if it has $n$ linearly
            independent eigenvectors.
        \end{theorem}\pause
        This means that if $A=CDC^{-1}$, then we have that 
        \[
            C = \bMat{\vec{v}_1 & \vec{v}_2 &  \dots & \vec{v}_n}\qquad D = \bMat{
                \lambda_1 & 0 & \dots & 0\\
                0 & \lambda_2 & \ddots & 0\\
                \vdots & \ddots & \ddots & \vdots\\
                0 & 0 & \dots & \lambda_n
            }
        \]
        where $(\vec{v}_1,\lambda_1),\dots,(\vec{v}_n,\lambda_n)$ are each eigenpairs.\pause\\
        \begin{tcolorbox}
            Remember that eigenvectors associated with distinct eigenvalues are linearly independent.\pause\
            So, if $A$ has $n$ distinct eigenvalues, then it is diagonalizable!
        \end{tcolorbox}
    \end{frame}
    \begin{frame}{Diagonalization is not Unique $2\times 2$ Example}
        Consider $A = \bMat{1&2\\0&4}$. We see that for $V_1,V_2,D\in\R^{2\times 2}$ as given below,
        \[
            V_1 = \bMat{
                1 & 2\\
                0 & 3
            }, V_2 = \bMat{
                2 & 2 \\
                0 & 3
            }, D = \bMat{1 & 0\\0 & 4}
        \]\pause
        We have
        \[
            AV_1 = V_1D\qquad AV_2 = V_2D
        \]\pause
        But $V_1\neq V_2$.
    \end{frame}
    \begin{frame}{Diagonalization}
        In order to diagonalize a matrix, $A\in\R^{n\times n}$, we do the following
        \begin{enumerate}
            \pause\item Compute all eigenvalues $\lambda_1,\dots, \lambda_k$ \pause ($k$ may not be $n$)
            \pause\item Compute a basis for each $E(A,\lambda_\ell)$.
            \pause\item If the total number of basis vectors is less than $n$, then $A$ is not
                diagonalizable, Otherwise, continue
            \pause\item Now $\vec{v}_1,\dots,\vec{v}_n$ (Eigenspace basis vectors!) form the columns
                of $C$, and their associated eigenvalues form the diagonal of $D$.
        \end{enumerate}
    \end{frame}
    \begin{frame}{Diagonalization Example}
        Let $A=\bMat{
            2 & 0 & 2\\
            0 & 1 & 0\\
            2 & 0 & 5
        }$ which has characteristic polynomial $-\lambda^3+8\lambda^2-13\lambda+6$, which has roots
        $1,6$. So we compute a basis for $E(A,1)$ and $E(A,6)$. We first do $E(A,1)$.
        \[
            \bMat{
                2 & 0 & 2\\
                0 & 1 & 0\\
                2 & 0 & 5
            }\pause\xrightarrow{B=A-I}\bMat{
                1 & 0 & 2\\
                0 & 0 & 0\\
                2 & 0 & 4
            }\pause\xrightarrow{R_2\leftrightarrow R_3}\bMat{
                1 & 0 & 2\\
                2 & 0 & 4\\
                0 & 0 & 0
            }\pause\xrightarrow{R_2=R_2-2R_1}\bMat{
                1 & 0 & 2\\
                0 & 0 & 0\\
                0 & 0 & 0
            }
        \]
        Whose null space has a basis of $\setBasic{\bMat{-2\\0\\1}, \bMat{0\\1\\0}}$.
    \end{frame}
    \begin{frame}{Diagonalization Example Part 2}
        Now, we compute a basis for $E(A,6)$. For
        \[
            A = \bMat{
                2 & 0 & 2\\
                0 & 1 & 0\\
                2 & 0 & 5
            }
        \]\pause
        \[
            \bMat{
                2 & 0 & 2\\
                0 & 1 & 0\\
                2 & 0 & 5
            }\pause\xrightarrow{B=A-6I}\bMat{
                -4 & 0 & 2\\
                0 & -5 & 0\\
                2 & 0 & -1
            }\pause\xrightarrow{R_3=R_3-\frac{1}{2}R_1}\bMat{
                -4 & 0 & 2\\
                0 & -5 & 0\\
                0 & 0 & 0
            }\pause\xrightarrow[R_2=-\frac{R_2}{5}]{R_1=-\frac{R_1}{4}}\bMat{
                1 & 0 & \frac{1}{2}\\
                0 & 1 & 0\\
                0 & 0 & 0
            }
        \]
        whose null space has a basis of $\setBasic{\bMat{-\frac{1}{2}\\0\\1}}$.
    \end{frame}
    \begin{frame}{Diagonalization Example Part 3}
        So we can say that $A$ is similar to $D=\bMat{1 & 0 & 0\\0 & 1 & 0 \\ 0 & 0 & 6}$ with 
        $C=\bMat{
            -2 & 0 & -\frac{1}{2}\\
            0 & 1 & 0\\
            1 & 0 & 1
        }$\pause or
        \[
            \bMat{
                2 & 0 & 2\\
                0 & 1 & 0\\
                2 & 0 & 5
            }\bMat{
                -2 & 0 & -\frac{1}{2}\\
                0 & 1 & 0\\
                1 & 0 & 1
            } = \bMat{
                -2 & 0 & -\frac{1}{2}\\
                0 & 1 & 0\\
                1 & 0 & 1
            }\bMat{
                1 & 0 & 0\\
                0 & 1 & 0\\
                0 & 0 & 6
            }
        \]
    \end{frame}
    \begin{frame}{Multiplicity of a Root Review}
        Recall that in the context of polynomials the \bText{multiplicity} of a root is the number
        of times it is present in factored form.\pause
        \begin{example}
            For the polynomial $x^3 - 3x + 2$, we can factor it into
            \[
                (x-1)^2(x+2)
            \]
            So, $x=1$ is a root with multiplicity $2$ and $x=-2$ is a root with multiplicity $1$.
        \end{example}
    \end{frame}
    \begin{frame}{Eigenvalue Multiplicities}
        \begin{defn}
            Let $A\in\R^{n\times n}$ with $\lambda$ as an eigenvalue of $A$.

            \rText{Algebraic Multiplicity}: The \bText{algebraic multiplicity} of $\lambda$ is the
            multiplicity as a root of the characteristic polynomial of $A$.\pause\\
            \bText{Geometric Multiplicity}: The \bText{geometric multiplicity} of $\lambda$ is
            the dimension of it's eigenspace (or $\dim{E(A,\lambda)}$).
        \end{defn}
        \pause
        \begin{tcolorbox}
            Let $A\in\R^{n\times n}$ and $\lambda$ be an eigenvalue of $A$. Then
            \[
                1\leq(\textnormal{the geometric multiplicity of }\lambda)\leq(\textnormal{the algebraic multiplicity of }\lambda)
            \]
        \end{tcolorbox}
    \end{frame}
    \begin{frame}{Variant of Diagonalizability Theorem}
        \begin{theorem}
            Let $A\in\R^{n\times  n}$. The following are equivalent:
            \begin{enumerate}
                \pause\item $A$ is diagonalizable
                \pause\item The sum of the geometric multiplicities of all eigenvalues of $A$ is equal to $n$.
            \end{enumerate}
        \end{theorem}
    \end{frame}
    \begin{frame}{Finding Multiplicity Example}
        Let $A = \bMat{
            1 &-2 & 1\\
            0 &-3 & 2\\
            2 & 3 &-1
        }$. Find the algebraic and geometric multiplicities of $1$.
        \pause

        The characteristic polynomial of $A$ is
        \[
            -\lambda^3 - 3\lambda^2 + 9\lambda -5\pause = -(\lambda-1)^2(\lambda+5)
        \]\pause
        So the algebraic multiplicity of $1$ is $2$.\pause\\
        We now row reduce $A-I$\pause
        \[
            \bMat{
                0 &-2 & 1\\
                0 &-4 & 2\\
                2 & 3 &-2
            }\pause\xrightarrow{R_1\leftrightarrow R_3}\bMat{
                2 & 3 &-2\\
                0 &-4 & 2\\
                0 &-2 & 1
            }\pause\xrightarrow{R_3=R_3-\frac{1}{2}R_2}\bMat{
                2 & 3 &-2\\
                0 &-4 & 2\\
                0 & 0 & 0

            }
        \]\pause
        There are $2$ pivot variables, so $\dim{E(A,1)}=2$ 


    \end{frame}
    \begin{frame}{Finding Multiplicities Practice}
        Let $A=\bMat{1&1\\0&1}$. It has only $1$ as an eigenvalue. Compute the algebraic
        and geometric multiplicities of $1$.
    \end{frame}
    \begin{frame}{Multiplicities for Similar Matrices}
        \begin{theorem}
            Let $A,B\in\R^{n\times n}$ such that $A$ and $B$ are similar \textbf{and} $\lambda$ be an eigenvalue of both $A,B$. Then:
            \begin{enumerate}
                \pause\item The algebraic multiplicity of $\lambda$ is the same for $A$ and
                    $B$.
                \pause\item The geometric multiplicity of $\lambda$ is the same for $A$ and
                    $B$.
            \end{enumerate}
        \end{theorem}
    \end{frame}
\end{document}
