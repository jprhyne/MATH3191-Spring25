\documentclass[xcoler=dvipsnames, aspectratio=169]{beamer}

\usepackage{3191Style}
% Date gives the title of the lecture
\date{Characteristic Polynomial}

\begin{document}
    \begin{frame}{Eigenpair Reminder}
        Remember that we defined an eigenpair of $A$ as an ordered pair $(\lambda,\vec{v})$ such that
        \[
            A\vec{v} = \lambda\vec{v}
        \]\pause
        We've discussed how to confirm a proposed $\lambda$ is actually an eigenvalue and to compute
        associated eigenvectors.\pause\ 

        What about how to compute the eigenvalues themselves?
    \end{frame}
    \begin{frame}{Computing Eigenvalues}
        Recall that if $\lambda$ is an eigenvalue of a matrix $A$, then 
        \[
            (A-\lambda I)\vec{v} = \vec{0}
        \]
        has a non-trivial solution.\pause\ In other words, $A-\lambda I$ is not invertible!\pause\

        So, we can figure out all $\lambda$ values such that
        \[
            \det{A-\lambda I} = 0
        \]
    \end{frame}
    \begin{frame}{Characteristic Polynomial}
        \small
        \begin{defn}
            Let $A\in\R^{n\times n}$. The \bText{characteristic polynomial} of $A$ is the function
            $f(\lambda)$ given by
            \[
                f(\lambda) = \det{A-\lambda I}
            \]
        \end{defn}\pause
        \begin{theorem}
            The roots of $f(\lambda)$ are exactly the eigenvalues of $A$.
        \end{theorem}\pause
        \begin{proof}
            Let $\lambda$ be an eigenvalue of $A$, then $A-\lambda I$ is not invertible, so
            $\det{A-\lambda I}=0$, so $f(\lambda) = \det{A-\lambda I} = 0$.\pause\

            Let $\lambda$ be a root of $f(\lambda)$, then
            \[
                0 = f(\lambda) = \det{A-\lambda I}.
            \]
            So, $A-\lambda I$ is not invertible, so $\lambda$ is an eigenvalue of $A$.
        \end{proof}
    \end{frame}
    \begin{frame}{Rational Root Theorem\footnote{A proof can be found at 
        \url{https://en.wikipedia.org/wiki/Rational_root_theorem}}}
        \footnotesize
        \begin{theorem}
            Let $f(\lambda) = c_0 + c_1\lambda + \dots + c_{n-1}\lambda^{n-1} + c_n\lambda^n$ 
            be a polynomial with integer coefficients.\pause\ Then, all rational factors of $f$ 
            are of the form
            \[
                x = \frac{p}{q}
            \]
            Where $p$ is an integer factor of $c_0$ and $q$ is an integer factor of $c_n$.
        \end{theorem}\pause
        \begin{example}
            Consider $f(x) = x^3+x^2-10x+8$. Then our possible roots are
            \[
                \setBasic{\pm 1, \pm 2, \pm 4, \pm 8}
            \]
            And we plug in these values to see which one(s) are actual roots. 
        \end{example}
        Note: We may not have rational roots depending on the polynomial itself (complex roots!)
    \end{frame}
    \begin{frame}{Finding Eigenvalues Example}
        \small
        Let $A\in\R^{3\times 3}$ as given below. Then compute all the eigenvalues of $A$.
        \[
            A = \bMat{
                4 & 1 &-1\\
                1 & 4 &-3\\
                -4&-4 & 5
            }
        \]\pause
        \[
            \det{A-\lambda I} = \det{\bMat{
                4-\lambda & 1 &-1\\
                1 & 4-\lambda &-3\\
                -4&-4 & 5-\lambda
            }} = -\lambda^3 +13\lambda^2 -39\lambda + 27 
        \]\pause
        Our rational roots theorem states that the possible rational eigenvalues are:
        \[
            \setBasic{\pm 1, \pm 3, \pm 9, \pm 27}
        \]
        So, we plug these in and find that $1,3,9$ are the eigenvalues.
    \end{frame}
    \begin{frame}{Finding Eigenvalues Practice}
        Let $A\in\R^{2\times 2}$ as given below. Then compute all the eigenvalues of $A$.
        \[
            A = \bMat{
                5 & -3\\
                3 & -5
            }
        \]
        \iftoggle{showSolutions}{
            \pause
            \[
                f(\lambda) = \lambda^2 - 16
            \]
            which has roots\pause
            \[
                \lambda=\pm 4
            \]
        }{\vspace{130pt}}
    \end{frame}
    \begin{frame}{Trace of a Matrix}
        \begin{defn}
            Let $A\in\R^{n\times n}$. Then we define the \bText{trace} of $A$ as
            \[
                \trace{A} = a_{11} + \dots + a_{nn}
            \]
            where
            \[
                A = \bMat{
                    \rText{a_{11}} & a_{12} & \dots & a_{1n}\\
                    a_{21} & \rText{a_{22}} & \dots & a_{2n}\\
                    \vdots & \vdots & \ddots& \vdots\\
                    a_{n1} & a_{n2} & \dots & \rText{a_{nn}}
                }
            \]
        \end{defn}
    \end{frame}
    \begin{frame}{Properties of the characteristic polynomial}
        Let $A\in\R^{n\times n}$. Then we know that 
        \[
            f(\lambda) = (-1)^n\lambda^n + (-1)^{n-1}\trace{A}\lambda^{n-1} + \dots + \det{A}
        \]\pause
        This means if $A\in\R^{2\times 2}$ of the form
        \[
            A = \bMat{
                a_{11} & a_{12}\\
                a_{21} & a_{22}
            }
        \]\pause\ 
        then the characteristic polynomial has the form:\pause
        \[
            f(\lambda) = \lambda^2 + (a_{11}+a_{22})\lambda + (a_{11}a_{22} - a_{12}a_{21})
        \]

    \end{frame}
    \begin{frame}{What About Larger Problems? (IE $n\geq 5$)}
        It's great that we can solve these small problems by hand, but what about larger ones?\pause\
        Well, The Abel–Ruffini theorem\footnote{More information can be found here: 
        \url{https://en.wikipedia.org/wiki/Abel\%E2\%80\%93Ruffini_theorem}} 
        states that we cannot always solve for these eigenvalues when we have $n\geq 5$.\pause\

        However, we can still try to solve these problems! We will achieve this via some methods
        we learn later in the semester, and is how we currently compute eigenvalues in practice.
    \end{frame}
\end{document}
