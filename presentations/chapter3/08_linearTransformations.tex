\documentclass[xcoler=dvipsnames, aspectratio=169]{beamer}
\usepackage{3191Style}
\date{Linear Transformations}
\begin{document}
\begin{frame}{Interpreting $A\vec{x}=\vec{b}$}
    The equation $A\vec{x} = \vec{b}$ conceptually is:
    \pause
    \begin{itemize}
        \item Finding some input $\vec{x}$ to give us an output $\vec{b}$\pause
        \item Making sure $A$ has enough information!\pause
    \end{itemize}
    \begin{example}
        Some mappings we know from this class or previous are:
        \begin{columns}
            \column{.5\textwidth}
            \begin{itemize}
                \item $f:x\mapsto x^2$ maps from $\R\rightarrow\R$.\pause
                \item $f:x\mapsto 7x$ maps from $\R\rightarrow\R$.\pause
                \item $f:\begin{bmatrix}x_1\\x_2\end{bmatrix}\mapsto 
                        \begin{bmatrix}7x_1\\7x_2\end{bmatrix}$ maps from $\R^2\rightarrow\R^2$.\pause
            \end{itemize}
            \column{.5\textwidth}
            \begin{itemize}
                \item $f:\begin{bmatrix}x_1\\x_2\end{bmatrix}\mapsto 
                        \begin{bmatrix}x_2\\x_1\\x_1+x_2\end{bmatrix}$ maps from
                            $\R^2\rightarrow\R^3$.\pause
                    \item $T:\vec{x}\mapsto A\vec{x}$ maps from $\R^n\rightarrow\R^m$.
            \end{itemize}
        \end{columns}
    \end{example}
\end{frame}
\begin{frame}{Transformations from $\R^n$ to $\R^m$}
    \begin{defn}
        \rText{Transformation}: A \bText{transformation} (or a function/mapping) $T$ from $\R^n$ to
        $\R^m$ is just a rule that assigns each vector $\vec{x}\in\R^n$ to another vector 
        $T(\vec{x})\in\R^m$.\pause
        \begin{itemize}
            \item All possible inputs, $\R^n$ is the \bText{domain} of $T$.\pause
            \item The set where the mappings will live, $\R^m$ is called the \bText{codomain} of $T$.\pause
            \item For each $\vec{x}$ in the domain, $T(\vec{x})\in\R^m$ is called the \bText{image}
                of $\vec{x}$.\pause
            \item All possible images (or outputs) is called the \bText{range} of $T$.
        \end{itemize}
    \end{defn}
    \vspace{-5pt}
    \begin{example}
        Consider
        \[
            T(\vec{x}) = A\vec{x} = \begin{bmatrix}1&0\\0&0\\0&1\end{bmatrix}\begin{bmatrix}
                x_1\\x_2\end{bmatrix}\pause = \begin{bmatrix}x_1\\0\\x_2\end{bmatrix}
        \]
    \end{example}
\end{frame}
\begin{frame}{Linear Transformations!}
    \begin{defn}
        \rText{Linear Transformation}: A transformation $T:\R^n\rightarrow\R^m$ is a 
        \bText{linear transformation} if and only if 
        \begin{itemize}
            \item for all $\vec{u},\vec{v}\in\R^n$, $T(\vec{u}+\vec{v}) = T(\vec{u}) + T(\vec{v})$
            \item for all $\vec{v}\in\R^n$ and scalars $c$, $T(c\vec{v}) = cT(\vec{v})$
        \end{itemize}
    \end{defn}
    \pause
    \vspace{60pt}
    \begin{theorem}
        Every matrix equation is a linear transformation, and vice versa (but only for 
        $T:\R^n\rightarrow\R^m$!)
    \end{theorem}
\end{frame}
\begin{frame}{Image of the Zero Vector}
    \begin{theorem}
        If $T:\R^n\rightarrow\R^m$ is a linear map, then $T(\vec{0}) = \vec{0}$.
    \end{theorem}
    \iftoggle{showSolutions}{
        \vspace{50pt}
        \begin{proof}
            Let $\vec{x}\in\R^n$ and $T(\vec{x})=\vec{b}\in\R^m$. See that\pause
            \[
                T(\vec{0}) = T(\vec{x}-\vec{x})\pause = T(\vec{x}) - T(\vec{x})\pause = 
                \vec{b}-\vec{b}\pause =\vec{0}
            \]
        \end{proof}
    }{
        \vspace{150pt}
    }
\end{frame}
\begin{frame}{Representing Transformations with Matrices}
    \begin{theorem}
        For all $T:\R^n\rightarrow\R^m$ there exists a unique $A\in\R^{m\times n}$ such that
        \[
            T(\vec{x}) = A\vec{x}
        \]
        and is given by
        \[
            A = \begin{bmatrix}T(\vec{e}_1) & \dots & T(\vec{e}_n)\end{bmatrix}
        \]
    \end{theorem}
    \vspace{150pt}
\end{frame}
\begin{frame}{Example}
    Define $T:\R^2\rightarrow\R^3$ such that
    \[
        T\left(\begin{bmatrix}x_1\\x_2\end{bmatrix}\right) = \begin{bmatrix}x_2\\x_1\\x_1+x_2\end{bmatrix}
    \]
    Find the associated matrix $A$ for this transformation.\pause\ 
    See that\pause
    \[
        T(\vec{e}_1) = \begin{bmatrix}0\\1\\1\end{bmatrix}\pause\qquad T(\vec{e_2}) = \begin{bmatrix}
            1\\0\\1\end{bmatrix}
    \]
    So, \pause
    \[
        A = \begin{bmatrix}
            0&1\\
            1&0\\
            1&1
        \end{bmatrix}
    \]
\end{frame}
\begin{frame}{Another Example}
    Let $T:\R^2\rightarrow\R^3$ such that
    \[
        T\left(\begin{bmatrix}1\\4\end{bmatrix}\right) = \begin{bmatrix}6\\8\\16\end{bmatrix}\qquad\textnormal{ and }
            T\left(\begin{bmatrix}2\\4\end{bmatrix}\right) = \begin{bmatrix}8\\8\\20\end{bmatrix}
    \]
    Construct the $A$ matrix associated with $T$.
\end{frame}
\begin{frame}{Another Example continued}
    Need to be able to find $T(\vec{e}_1)$ and $T(\vec{e}_2)$!\pause

    So, we write $\vec{e}_1$ and $\vec{e}_2$ in terms of our given vectors. So we reduce:

    \[
        \aMat{cc|c}{
            1&2&b_1\\
            4&4&b_2
        }\pause\rightarrow\aMat{cc|c}{
            1&2&b_1\\
            0&-4&b_2-4b_1
        }\pause\rightarrow\aMat{cc|c}{
            1&2&b_1\\
            0&1&b_1-\frac{b_2}{4}
        }\pause\rightarrow\aMat{cc|c}{
            1&0&\frac{b_2}{2}-b_1\\
            0&1&b_1-\frac{b_2}{4}
        }
    \]
    So:\pause
    \[
        \vec{e}_1 = -1\begin{bmatrix}1\\4\end{bmatrix}+1\begin{bmatrix}2\\4\end{bmatrix}
            \qquad\vec{e}_2 = \frac{1}{2}\begin{bmatrix}1\\4\end{bmatrix}+
                -\frac{1}{4}\begin{bmatrix}2\\4\end{bmatrix}
    \]
\end{frame}
\begin{frame}{Another Example continued}
    \[
        T(\vec{e}_1) = T\left(-1\begin{bmatrix}1\\4\end{bmatrix}+1\begin{bmatrix}2\\4\end{bmatrix}\right)
            \pause = -T\left(\begin{bmatrix}1\\4\end{bmatrix}\right) + 
            T\left(\begin{bmatrix}2\\4\end{bmatrix}\right)\pause = 
            -\begin{bmatrix}6\\8\\16\end{bmatrix} + \begin{bmatrix}8\\8\\20\end{bmatrix}
            \pause = \begin{bmatrix}2\\0\\4\end{bmatrix}\pause
    \]
    and
    \[
        T(\vec{e}_2) = T\left(\frac{1}{2}\begin{bmatrix}1\\4\end{bmatrix}-\frac{1}{4}\begin{bmatrix}2\\4\end{bmatrix}\right)
            \pause = \frac{1}{2}T\left(\begin{bmatrix}1\\4\end{bmatrix}\right) -\frac{1}{4} 
            T\left(\begin{bmatrix}2\\4\end{bmatrix}\right)\pause = 
            \frac{1}{2}\begin{bmatrix}6\\8\\16\end{bmatrix} -\frac{1}{4} \begin{bmatrix}8\\8\\20\end{bmatrix}
            \pause = \begin{bmatrix}1\\2\\3\end{bmatrix}\pause
    \]
    So:
    \[
        A = \begin{bmatrix}T(\vec{e}_1) & T(\vec{e}_2)\end{bmatrix}\pause = \begin{bmatrix}
            2 & 1\\
            0 & 2\\
            4 & 3
        \end{bmatrix}
    \]
\end{frame}
\begin{frame}{Now You Try!}
    \scriptsize
    Let $T:\R^2\rightarrow\R^2$ be such that
    \[
        T\left(\begin{bmatrix}1\\1\end{bmatrix}\right) = \begin{bmatrix}3\\1\end{bmatrix}\qquad
        T\left(\begin{bmatrix}2\\1\end{bmatrix}\right) = \begin{bmatrix}4\\1\end{bmatrix}
    \]
    Compute the $A$ associated with $T$.
    \iftoggle{showSolutions}{
        \pause
        \[
            \aMat{cc|c}{
                1&2&b_1\\
                1&1&b_2
            }\pause\rightarrow\aMat{cc|c}{
                1&2&b_1\\
                0&-1&b_2-b_1
            }\pause\rightarrow\aMat{cc|c}{
                1&2&b_1\\
                0&1&b_1-b_2
            }\pause\rightarrow\aMat{cc|c}{
                1&0&2b_2-b_1\\
                0&1&b_1-b_2
            }
        \]
        So, 
        \[
            T(\vec{e}_1) = -T\left(\begin{bmatrix}1\\1\end{bmatrix}\right) 
            + T\left(\begin{bmatrix}2\\1\end{bmatrix}\right)\pause= \begin{bmatrix}1\\0\end{bmatrix},\qquad
            T(\vec{e}_2) = 2T\left(\begin{bmatrix}1\\1\end{bmatrix}\right) 
            - T\left(\begin{bmatrix}2\\1\end{bmatrix}\right)\pause= \begin{bmatrix}2\\1\end{bmatrix}
        \]
        \pause\ And:
        \[
            A = \begin{bmatrix}1&2\\0&1\end{bmatrix}
        \]
    }{
        \vspace{150pt}
    }
\end{frame}
\begin{frame}{Properties of Linear Transformations}
    \begin{tcolorbox}
        Let $T:\R^n\rightarrow\R^m$ be a linear transformation The mapping $T$ is said to be:
        \begin{itemize}
            \item \rText{Injective} if for each $\vec{b}\in\R^m$, we can find \textbf{at most one} 
                $\vec{x}$ such that $T(\vec{x})=\vec{b}$\pause
            \item \bText{Surjective} if for each $\vec{b}\in\R^m$, we can find \textbf{at least one}
                $\vec{x}$ such that $T(\vec{x})=\vec{b}$
        \end{itemize}
    \end{tcolorbox}
        \vspace{150pt}
\end{frame}
\begin{frame}{Python Example}
    Consider $T: \R^4\rightarrow\R^4$, $\vec{x}\mapsto A\vec{x}$ for
    \[
        A = \begin{bmatrix}4 & -2 & 5 & -5 \\
            -9 & 7 & -8 & 0\\
            -6&4&5&3\\
            5&-3&8&-4
        \end{bmatrix}
    \]
    \begin{enumerate}
        \item Find all $\vec{x}$ such that $T(\vec{x}) = \vec{0}$
        \item Find all $\vec{x}$ (if any) such that $T(\vec{x}) = \begin{bmatrix}7\\5\\9\\7\end{bmatrix}$
    \end{enumerate}
\end{frame}
\begin{frame}{Geometric Interpretation in $\mathbb{R}^2$}

Consider the linear transformation given by $T: \mathbb{R}^2 \to \mathbb{R}^2: \mathbf{x} \mapsto 0.5 \mathbf{x}$. Find the standard matrix $A$ for this linear transformation. \bs

We have  $\dsty \alert{T(\mathbf{e}_1) = \begin{bmatrix} 0.5 \\ 0 \end{bmatrix}}$ and $\dsty \colorb{T(\mathbf{e}_2) = \begin{bmatrix} 0 \\ 0.5 \end{bmatrix}}$ giving the matrix $\dsty A = \begin{bmatrix} \alert{0.5} & \colorb{0} \\ \alert{0} & \colorb{0.5} \end{bmatrix}$.

\begin{columns}

\column{0.33\tw}

\begin{center}
\includegraphics[width=0.8\tw]{../Images/Chap1/fig-ele-pic.JPG}
\end{center}

\column{0.33\tw}

\begin{center}
\includegraphics[width=0.8\tw]{../Images/Chap1/fig-elephant.png}
\end{center}

\column{0.33\tw}

\begin{center}
\includegraphics[width=0.8\tw]{../Images/Chap1/fig-ele-dilate.png}
\end{center}

\end{columns}

\end{frame}

\begin{frame}{Geometric Interpretation in $\mathbb{R}^2$}

Consider the linear transformation given by $\dsty T(\mathbf{x}) = A \mathbf{x} = \begin{bmatrix} -1 & 0 \\ 0 & 1 \end{bmatrix}$ 

We have  $\dsty \alert{T(\mathbf{e}_1) = \begin{bmatrix} -1 \\ 0 \end{bmatrix}}$ and $\dsty \colorb{T(\mathbf{e}_2) = \begin{bmatrix} 0 \\ 1 \end{bmatrix}}$ giving the geometric interpretation seen below.

\begin{columns}

\column{0.5\tw}

\begin{center}
\includegraphics[width=0.6\tw]{../Images/Chap1/fig-elephant.png}
\end{center}

\column{0.5\tw}

\begin{center}
\includegraphics[width=0.6\tw]{../Images/Chap1/fig-ele-reflect.png}
\end{center}

\end{columns}

\end{frame}

\begin{frame}{Contractions and Expansions}

Consider the linear transformation given by $\dsty T(\mathbf{x}) = A \mathbf{x} = \begin{bmatrix} 1 & 0 \\ 0 & 2 \end{bmatrix}$ 

We have  $\dsty \alert{T(\mathbf{e}_1) = \begin{bmatrix} 1 \\ 0 \end{bmatrix}}$ and $\dsty \colorb{T(\mathbf{e}_2) = \begin{bmatrix} 0 \\ 2 \end{bmatrix}}$ giving the geometric interpretation seen below.

\begin{columns}

\column{0.5\tw}

\begin{center}
\includegraphics[width=0.6\tw]{../Images/Chap1/fig-elephant2.png}
\end{center}

\column{0.5\tw}

\begin{center}
\includegraphics[width=0.6\tw]{../Images/Chap1/fig-ele-expand.png}
\end{center}

\end{columns}

\end{frame}

\begin{frame}{Shear Transformations}

Consider the linear transformation given by $\dsty T(\mathbf{x}) = A \mathbf{x} = \begin{bmatrix} 1 & 0.75 \\ 0 & 1 \end{bmatrix}$ 


\begin{columns}

\column{0.5\tw}

\begin{center}
\includegraphics[width=0.6\tw]{../Images/Chap1/fig-elephant.png}
\end{center}

\column{0.5\tw}

\begin{center}
\includegraphics[width=0.6\tw]{../Images/Chap1/fig-ele-shear.png}
\end{center}

\end{columns}

\end{frame}

\begin{frame}{Projections}

Consider the linear transformation given by $\dsty T(\mathbf{x}) = A \mathbf{x} = \begin{bmatrix} 1 & 0 \\ 0 & 0 \end{bmatrix}$ 


\begin{columns}

\column{0.5\tw}

\begin{center}
\includegraphics[width=0.6\tw]{../Images/Chap1/fig-elephant.png}
\end{center}

\column{0.5\tw}

\begin{center}
\includegraphics[width=0.6\tw]{../Images/Chap1/fig-ele-project.png}
\end{center}

\end{columns}

\end{frame}
\begin{frame}
    \begin{theorem}
        A linear transformation $T:\R^n\rightarrow\R^m$ is injective if and only if $T(\vec{x})=\vec{0}$
        has only the trivial solution
    \end{theorem}
    \iftoggle{showSolutions}{
        \begin{proof}
            $\implies$ direction. Assume that $T$ is injective and there is some $\vec{v}\neq\vec{0}$
            such that $T(\vec{v}) = \vec{0}$. Well, this contradicts the fact that $T$ is injective
            because $T(\vec{0}) = T(\vec{v}) = \vec{0}$.\\
            $\impliedby$ direction. Assume $T$ is not injective. This means there are some
            $\vec{u}$ and $\vec{v}$ such that $\vec{u}\neq\vec{v}$, and $T(\vec{u})=T(\vec{v})=
            \vec{b}\in\R^m$ and $T(\vec{x}) = \vec{0}$ has only the trivial solution. See that
            \[
                T(\vec{u}-\vec{v}) = T(\vec{u}) - T(\vec{v}) = \vec{b}-\vec{b} = \vec{0}
            \]
            Since $\vec{u}\neq\vec{v}$, we know that $\vec{u}-\vec{v}\neq\vec{0}$. This contradicts
            the fact that $T(\vec{x})=\vec{0}$ has only the trivial solution.
        \end{proof}
    }{
        \vspace{150pt}
    }
\end{frame}
\begin{frame}{Summary}
    \begin{tcolorbox}
        A linear transformation $T:\R^n\rightarrow\R^m$ with associated matrix $A$ is \rText{injective}
        if and only if
        \begin{itemize}
            \item $T(\vec{x}) = \vec{0}$ has only the trivial solution.\pause
            \item The solution set to $A\vec{x} = \vec{0}$ has no free variables.\pause
            \item The matrix $A$ has a pivot in every column.\pause
            \item \rText{The columns of $A$ are linearly independent}.
        \end{itemize}
    \end{tcolorbox}
    \onslide<5->{
    \begin{tcolorbox}
        A linear transformation $T:\R^n\rightarrow\R^m$ with associated matrix $A$ is 
        \bText{surjective} if and only if
        \begin{itemize}
            \item For any $\vec{b}\in\R^m$, there exists at least one $\vec{x}\in\R^n$ such that
                $A\vec{x}=\vec{b}$.
            \onslide<6->{
            \item \bText{The columns of A span all of $\R^m$}.
            }
        \end{itemize}
    \end{tcolorbox}
    }
\end{frame}
\end{document}
