\documentclass[xcoler=dvipsnames, aspectratio=169]{beamer}
\usepackage{3191Style}
\date{Matrix Inverses}

\begin{document}
    \begin{frame}{The Inverse of a Map}
        \begin{tcolorbox}
            Recall that for a function $f$, it's \rText{inverse function} $f^{-1}$ is a function
            that ``undoes'' what $f$ did. Or in other words:\pause
            \begin{itemize}
                \item $f^{-1}(f(x)) = x$ for all $x$ in the domain of $f$.\pause
                \item $f(f^{-1}(y)) = y$ for all $x$ in the domain of $f^{-1}$.\pause
            \end{itemize}
        \end{tcolorbox}
        \only<4->{
            \begin{example}
                If $f(x) = 2x$, then $f^{-1}(x) = \pause\frac{x}{2}$.\pause

                If $f(x) = \frac{x}{5} - 7$, then $f^{-1}(x) = \pause (x+7)\cdot 5$.
            \end{example}
        }
    \end{frame}
    \begin{frame}{The Inverse of a Matrix}
        Let $A\in\R^{n\times n}$. The \rText{inverse matrix} if it exists, is denoted $A^{-1}$ and
        is the unique matrix such that:\pause
        \begin{itemize}
            \item $AA^{-1} = I_n$\pause
            \item $A^{-1}A = I_n$\pause
        \end{itemize}
        \begin{example}
            Let $A = \begin{bmatrix}1 & 0 \\ 0 & 5\end{bmatrix}$.
            \begin{itemize}
                \item What is $A$?\pause\ \bText{Vertical expansion} by factor of $5$.\pause
                \item How do we undo $A$?\pause\ \bText{Vertical compression} by factor of $5$.\pause
            \end{itemize}
            So:\pause
            \[
                A^{-1} = \begin{bmatrix}1 & 0\\0&\frac{1}{5}\end{bmatrix}
            \]
        \end{example}
    \end{frame}
    \begin{frame}{Elementary Matrices}
        \scriptsize
        \begin{defn}
            \rText{Elementary Matrices}: An \bText{elementary matrix} is one attainable by performing
            \textbf{one} row operation on an identity matrix.
        \end{defn}
        \pause
        \begin{example}
            \[
                E_1 = \begin{bmatrix}1&0\\0&5\end{bmatrix}\pause, E_2 = \begin{bmatrix}1&0&0\\0&1&-2\\0&0&1\end{bmatrix}
                \pause, E_3 = \begin{bmatrix}
                    1&0&0&0\\
                    0&0&0&1\\
                    0&0&1&0\\
                    0&1&0&0
                \end{bmatrix}
            \]
        \end{example}
        We can also find $E^{-1}_2$ and $E^{-1}_3$!
        \pause
        \begin{example}
            \[
                E^{-1}_2 = \begin{bmatrix}1&0&0\\0&1&2\\0&0&1\end{bmatrix}
                    \pause, E^{-1}_3 = \begin{bmatrix}
                    1&0&0&0\\
                    0&0&0&1\\
                    0&0&1&0\\
                    0&1&0&0
                \end{bmatrix}
            \]
        \end{example}
    \end{frame}
    \begin{frame}{Demonstrating $E^{-1}_2$}
        Recall
        \[
            E_2 = \begin{bmatrix}1&0&0\\0&1&-2\\0&0&1\end{bmatrix}, 
            E^{-1}_2 = \begin{bmatrix}1&0&0\\0&1&2\\0&0&1\end{bmatrix}
        \]
        \pause
        \[
            E^{-1}_2E_2 = \begin{bmatrix}1&0&0\\0&1&2\\0&0&1\end{bmatrix}\begin{bmatrix}1&0&0\\0&1&-2\\0&0&1\end{bmatrix} 
                \pause = \begin{bmatrix}1\\0\\0\end{bmatrix}\begin{bmatrix}1&0&0\end{bmatrix} + \begin{bmatrix}0\\1\\0\end{bmatrix}
                    \begin{bmatrix}0&1&-2\end{bmatrix} + \begin{bmatrix}0\\2\\1\end{bmatrix}\begin{bmatrix}0&0&1\end{bmatrix}\pause
        \]
        \[
            =\begin{bmatrix}1&0&0\\0&0&0\\0&0&0\end{bmatrix} + \begin{bmatrix}0&0&0\\0&1&-2\\0&0&0\end{bmatrix}
                + \begin{bmatrix}0&0&0\\0&0&2\\0&0&1\end{bmatrix}\pause = \begin{bmatrix}1&0&0\\0&1&0\\0&0&1\end{bmatrix}
                    \pause\checkmark
        \]
        Are we done?\pause\ No! Need to show $E_2E_2^{-1}=I_3$!
    \end{frame}
    \begin{frame}{Showing $E_2E_2^{-1}=I_3$}
        Your turn!
        \[
            E_2 = \begin{bmatrix}1&0&0\\0&1&-2\\0&0&1\end{bmatrix}, 
            E^{-1}_2 = \begin{bmatrix}1&0&0\\0&1&2\\0&0&1\end{bmatrix}
        \]
        \iftoggle{showSolutions}{
        \[
            E_2E^{-1}_2 = \begin{bmatrix}1&0&0\\0&1&-2\\0&0&1\end{bmatrix}\begin{bmatrix}1&0&0\\0&1&2\\0&0&1\end{bmatrix} 
                \pause = \begin{bmatrix}1\\0\\0\end{bmatrix}\begin{bmatrix}1&0&0\end{bmatrix} + \begin{bmatrix}0\\1\\0\end{bmatrix}
                    \begin{bmatrix}0&1&2\end{bmatrix} + \begin{bmatrix}0\\-2\\1\end{bmatrix}\begin{bmatrix}0&0&1\end{bmatrix}\pause
        \]
        \[
            =\begin{bmatrix}1&0&0\\0&0&0\\0&0&0\end{bmatrix} + \begin{bmatrix}0&0&0\\0&1&2\\0&0&0\end{bmatrix}
                + \begin{bmatrix}0&0&0\\0&0&-2\\0&0&1\end{bmatrix}\pause = \begin{bmatrix}1&0&0\\0&1&0\\0&0&1\end{bmatrix}
                    \pause\checkmark
        \]
        }{\vspace{130pt}}
    \end{frame}
    \begin{frame}{Inverse Properties}
        \begin{tcolorbox}
            Let $A,B\in\R^{n\times n}$ such that both are invertible. Then we know that\pause
            \begin{itemize}
                \item \rTextWait{$A^{-1}$ is invertible and $\left(A^{-1}\right)^{-1} = A$}{4-7}\pause
                \item \bTextWait{$AB$ is invertible and $(AB)^{-1}=B^{-1}A^{-1}$}{8-}\pause
            \end{itemize}
        \end{tcolorbox}
        \iftoggle{showSolutions}{
            \only<1-3>{\vspace{130pt}}
            \only<4->{
                \begin{columns}
                    \column{.5\textwidth}
                    \rTextWait{
                        \[
                             A^{-1}A \pause = I_n\pause
                        \]
                        \[
                            AA^{-1\pause} = I_n\pause
                        \]
                    }{4-7}
                    \column{.5\textwidth}
                    \bTextWait{
                        \[
                            (AB)B^{-1}A^{-1}\pause = A(BB^{-1})A^{-1}\pause = A(I_n)A^{-1}\pause
                        \]
                        \vspace{-20pt}
                        \[
                            = (AI_n)A^{-1}\pause = AA^{-1}\pause = I_n\pause
                        \]
                        \[
                            B^{-1}A^{-1}(AB)\pause = B(AA^{-1})B^{-1}\pause = B(I_n)B^{-1}\pause
                        \]
                        \vspace{-20pt}
                        \[
                            = (BI_n)B^{-1}\pause = BB^{-1}\pause = I_n
                        \]
                    }{8-}
                \end{columns}
                \vspace{50pt}
            }
        }{\vspace{130pt}}
    \end{frame}
    \begin{frame}{More Inverse Properties}
        \begin{tcolorbox}
            Let $A\in\R^{n\times n}$, then if $A$ is invertible, so is $A^\top$ and \pause
            \[
                \left(A^\top\right)^{-1} = \left(A^{-1}\right)^\top
            \]
        \end{tcolorbox}
        \pause
        \only<1-2>{\vspace{130pt}}
        \only<3->{
            \[
                A^\top\left(A^{-1}\right)^\top\pause = (A)^\top\left(A^{-1}\right)^\top\pause = 
                \left(A^{-1}A\right)^\top\pause = I_n^\top\pause = I_n\pause
            \]
            \[
                \left(A^{-1}\right)^\top A^\top\pause = \left(A^{-1}\right)^\top (A)^\top\pause =
                \left(AA^{-1}\right)^\top\pause = I_n^\top\pause = I_n
            \]
            \vspace{100pt}
        }
    \end{frame}
    \begin{frame}{Product of Elementary Matrices}
        \small
        Let $A = \begin{bmatrix}0&1&-2\\1&0&0\\0&0&5\end{bmatrix}$. Write $A$ as a product of
            elementary matrices.\pause
            \begin{columns}
                \column{.25\textwidth}
                    \begin{enumerate}
                        \item $R_2 = R_2 - 2R_3$
                        \only<4->{\item Swap $R_1$ and $R_2$}
                        \only<6->{\item $R_3 = 5R_3$.}
                    \end{enumerate}
                \column{.75\textwidth}
                \[
                    \only<3->{
                    E_1 = \begin{bmatrix}
                        1&0&0\\
                        0&1&-2\\
                        0&0&1
                    \end{bmatrix},} 
                    \only<5->{E_2 = \begin{bmatrix}
                        0 & 1 & 0\\
                        1 & 0 & 0\\
                        0 & 0 & 1
                    \end{bmatrix},}
                    \only<7->{E_3 = \begin{bmatrix}
                        1&0&0\\
                        0&1&0\\
                        0&0&5
                    \end{bmatrix}}
                \]
            \end{columns}
            \pause\pause\pause\pause\pause\pause
            \[
                A=E_3E_2E_1\pause =\begin{bmatrix}
                        1&0&0\\
                        0&1&0\\
                        0&0&5
                    \end{bmatrix}\begin{bmatrix}
                        0 & 1 & 0\\
                        1 & 0 & 0\\
                        0 & 0 & 1
                    \end{bmatrix}\begin{bmatrix}
                        1&0&0\\
                        0&1&-2\\
                        0&0&1
                    \end{bmatrix}\pause = \begin{bmatrix}
                        0&1&-2\\
                        1&0&0\\
                        0&0&5
                    \end{bmatrix}
            \]
            \pause
            \[
                A^{-1} = E_1^{-1}E_2^{-1}E^{-1}_3\pause = \begin{bmatrix}
                        1&0&0\\
                        0&1&2\\
                        0&0&1
                    \end{bmatrix}\pause\begin{bmatrix}
                        0 & 1 & 0\\
                        1 & 0 & 0\\
                        0 & 0 & 1
                    \end{bmatrix}\pause\begin{bmatrix}
                        1&0&0\\
                        0&1&0\\
                        0&0&\frac{1}{5}
                    \end{bmatrix}\pause = \begin{bmatrix}
                        0 & 1 & 0\\
                        1 & 0 & \frac{2}{5}\\
                        0 & 0 & \frac{1}{5}
                    \end{bmatrix}
            \]
    \end{frame}
    \begin{frame}{How to Compute $A^{-1}$ in General?}
        \small
        We can perform \rText{Gaussian Elimination} on $\aMat{c|c}{A&I_n}$!\pause
        
        If $A$ is \bText{row equivalent} to $I_n$ (IE we reduce the left part to $I_n$), then\pause
        \begin{itemize}
            \item $\aMat{c|c}{A&I_n}$ is row equivalent to $\aMat{c|c}{I_n&A^{-1}}$\pause
            \item Otherwise, $A^{-1}$ doesn't exist!\pause
        \end{itemize}
        \begin{example}
            \[
                \aMat{c|c}{A&I_n}\pause = \aMat{ccc|ccc}{
                    0 & 1 & -2 & 1 & 0 & 0 \\
                    1 & 0 & 0 &  0 & 1 & 0 \\
                    0 & 0 & 5 &  0 & 0 & 1
                }\pause\rightarrow\aMat{ccc|ccc}{
                    1 & 0 & 0 &  0 & 1 & 0 \\
                    0 & 1 & -2 & 1 & 0 & 0 \\
                    0 & 0 & 5 &  0 & 0 & 1
                }\pause\rightarrow\aMat{ccc|ccc}{
                    1 & 0 & 0 &  0 & 1 & 0 \\
                    0 & 1 & -2 & 1 & 0 & 0 \\
                    0 & 0 & 1 &  0 & 0 & \frac{1}{5}
                }\pause
            \]
            \[\rightarrow\aMat{ccc|ccc}{
                    1 & 0 & 0 &  0 & 1 & 0 \\
                    0 & 1 & 0 &  1 & 0 & \frac{2}{5} \\
                    0 & 0 & 1 &  0 & 0 & \frac{1}{5}
                }
            \]
        \end{example}
    \end{frame}
    \begin{frame}{How to Compute $A^{-1}$ Practice $3\times 3$}
        If possible, find the inverse of the given matrices.
        \begin{columns}
            \column{.5\textwidth}
            \[
                \begin{bmatrix}
                    1 & 0 & 3\\
                    0 & -2&-2\\
                    1 & -3& 1
                \end{bmatrix}
            \]
            \column{.5\textwidth}
            \[
                \begin{bmatrix}
                    2 & 0 & 1\\
                    0 & 1 & 3\\
                   -4 & 0 &-2
                \end{bmatrix}
            \]
        \end{columns}
        \vspace{130pt}
    \end{frame}
    \iftoggle{showSolutions}{
        \begin{frame}{$A^{-1}$ Solution 1}
            \scriptsize
            \[
                A = \begin{bmatrix}
                        1 & 0 & 3\\
                        0 & -2&-2\\
                        1 & -3& 1
                    \end{bmatrix}
            \]
            \pause
            \[
                \aMat{ccc|ccc}{
                    1 & 0 & 3 & 1 & 0 & 0\\
                    0 & -2&-2 & 0 & 1 & 0\\
                    1 & -3& 1 & 0 & 0 & 1
                }\xrightarrow{R_3=R_3-R1}\aMat{ccc|ccc}{
                    1 & 0 & 3 & 1 & 0 & 0\\
                    0 & -2&-2 & 0 & 1 & 0\\
                    0 & -3& -2& -1& 0 & 1
                }\pause\xrightarrow{R_2=-\frac{R_2}{2}}\aMat{ccc|ccc}{
                    1 & 0 & 3 & 1 & 0 & 0\\
                    0 & 1 & 1 & 0 & -\frac{1}{2} & 0\\
                    0 & -3& -2& -1& 0 & 1
                }
            \]
            \[
                \pause\xrightarrow{R_3=R_3+3R_2}\aMat{ccc|ccc}{
                    1 & 0 & 3 & 1 & 0 & 0\\
                    0 & 1 & 1 & 0 & -\frac{1}{2} & 0\\
                    0 & 0 & 1& -1& -\frac{3}{2} & 1
                }\pause\xrightarrow{R_2=R_2-R_3}\aMat{ccc|ccc}{
                    1 & 0 & 3 & 1 & 0 & 0\\
                    0 & 1 & 0 & 1 & 1 & -1\\
                    0 & 0 & 1& -1& -\frac{3}{2} & 1
                }
            \]
            \[
                \pause\xrightarrow{R_1-3R_3}\aMat{ccc|ccc}{
                    1 & 0 & 0 & 4 & \frac{9}{2} & -3\\
                    0 & 1 & 0 & 1 & 1 & -1\\
                    0 & 0 & 1& -1& -\frac{3}{2} & 1
                }
            \]
        \end{frame}
        \begin{frame}{$A^{-1}$ Solution 2}
            \[
                A = \begin{bmatrix}
                    2 &0& 1\\
                    0 &1& 3\\
                    -4&0&-2
                \end{bmatrix}
            \]
            \[
                \aMat{ccc|ccc}{
                    2 &0& 1&1&0&0\\
                    0 &1& 3&0&1&0\\
                    -4&0&-2&0&0&1
                }\xrightarrow{R_3=R_3+2R_1}\aMat{ccc|ccc}{
                    2 &0& 1&1&0&0\\
                    0 &1& 3&0&1&0\\
                    \rTextWait{0}{2-}&\rTextWait{0}{2-}&\rTextWait{0}{2-}&\rTextWait{0}{2-}
                    &\rTextWait{0}{2-}&\rTextWait{1}{2-}
                }
            \]
            \pause
            \rText{Can't find the inverse!}
        \end{frame}
    }{}
    \begin{frame}{How to Compute $A^{-1}$ Practice $2\times 2$}
        If possible, find the inverse of the given matrices.
        \begin{columns}
            \column{.5\textwidth}
            \[
                \begin{bmatrix}
                    3 & 6 \\
                    1 & 2
                \end{bmatrix}
            \]
            \column{.5\textwidth}
            \[
                \begin{bmatrix}
                   -2 & 4 \\
                   -3 & 1
                \end{bmatrix}
            \]
        \end{columns}
        \vspace{130pt}
    \end{frame}
    \iftoggle{showSolutions}{
        \begin{frame}{$A^{-1}$ solution 3}
            \[
                A = \begin{bmatrix}
                        3 & 6 \\
                        1 & 2
                    \end{bmatrix}
            \]
            \[
                \aMat{cc|cc}{
                    3&6&1&0\\1&2&0&1
                }\xrightarrow{R_2=R_2-\frac{R_1}{3}}\aMat{cc|cc}{
                    3&6&1&0\\
                    \rTextWait{0}{2-}&\rTextWait{0}{2-}&\rTextWait{-\frac{1}{3}}{2-}&\rTextWait{1}{2-}
                }
            \]
            \pause
            \rText{Can't find the inverse!}
        \end{frame}
        \begin{frame}{$A^{-1}$ solution 4}
            \[
                A = \begin{bmatrix}
                   -2 & 4 \\
                   -3 & 1
                \end{bmatrix}
            \]
            \[
                \aMat{cc|cc}{
                    -2 & 4 & 1 & 0\\
                    -3 & 1 & 0 & 1
                }\xrightarrow{R_2=R_2-\frac{3R_1}{2}}\aMat{cc|cc}{
                    -2 & 4 & 1 & 0\\
                    0 & -5 & -\frac{3}{2} & 1
                }\pause\xrightarrow[R_2=\frac{-R_2}{5}]{R_1=\frac{-R_1}{2}}\aMat{cc|cc}{
                    1 & -2 & -\frac{1}{2} & 0\\
                    0 &  1 & \frac{3}{10} & -\frac{1}{5}
                }
            \]
            \[
                \pause\xrightarrow{R_1=R_1+2R_2}\aMat{cc|cc}{
                    1 & 0 & \frac{1}{10} & -\frac{2}{5}\\
                    0 & 1 & \frac{3}{10} & -\frac{1}{5}
                }
            \]
        \end{frame}
    }{}
    \begin{frame}{Solving a System With $A^{-1}$}
        \small
        Let's say we know $A^{-1}$. How can we solve linear systems like $A\vec{x} = \vec{b}$?\pause

        \[
            A\vec{x} = \vec{b} \pause\rightarrow A^{-1}A\vec{x} = A^{-1}\vec{b} \pause\rightarrow \vec{x} = A^{-1}\vec{b}
        \]
        \pause
        This gives us our potential method as\pause
        \begin{tcolorbox}
        \begin{enumerate}
            \item Write our system as $A\vec{x} = \vec{b}$.\pause
            \item find $A^{-1}$\pause
                \begin{enumerate}
                    \item If we cannot, then we must use GE as normal\pause
                    \item If we can, then continue\pause
                \end{enumerate}
            \item Multiply both sides of (1) by $A^{-1}$.\pause
            \item Solution is $\vec{x} = A^{-1}\vec{b}$
        \end{enumerate}
        \end{tcolorbox}
        \begin{tcolorbox}
            Note: If $A$ is \textbf{NOT} square, then we cannot find an inverse!
        \end{tcolorbox}
    \end{frame}
\end{document}
