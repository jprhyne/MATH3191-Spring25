\documentclass[xcoler=dvipsnames, aspectratio=169]{beamer}
\usepackage{3191Style}
\date{Invertible Matrix Theorem}
\begin{document}
    \begin{frame}{Theorem Statement}
        \begin{theorem}
            Let $T:\R^n\rightarrow\R^n$ be a linear transformation with associated matrix 
            $A\in\R^{n\times n}$.
            \begin{columns}
                \column{.5\textwidth}
                \begin{enumerate}
                    \item $A$ is invertible
                    \pause
                    \item $A$ has a pivot in every row
                    \item $A$ has a pivot in every column
                    \pause
                    \item The system $A\vec{x}=\vec{b}$ has a unique solution for all 
                        $\vec{b}\in\R^n$
                    \pause
                    \item $A\vec{x}=\vec{0}$ has only the trivial solution
                    \pause
                \end{enumerate}
                \column{.5\textwidth}
                \begin{enumerate} \addtocounter{enumi}{5}
                    \item The columns of $A$ are linearly independent
                    \pause
                    \item The columns of $A$ span all of $\R^n$.
                    \pause
                    \item $T$ is invertible
                    \pause
                    \item $T$ is injective
                    \pause
                    \item $T$ is surjective
                \end{enumerate}
            \end{columns}
        \end{theorem}
        \pause
        We have the tools to prove all these, but really, it's just a summary of all the 
        linear transformation and matrix algebra we've done so far!
    \end{frame}

    \begin{frame}{One-sided inverse Implies the Other!}
        \begin{theorem}
            Let $A\in\R^{n\times n}$. If there is some $B\in\R^{n\times n}$ such that 
            \[
                BA = I_n\textnormal{ or }AB = I_n
            \]
            Then, $A$ is invertible and $A^{-1}=B$.
        \end{theorem}
    \end{frame}

    \begin{frame}{$AB=I_n$ case}
        \iftoggle{showSolutions}{
            \begin{proof}
                First, let $B\in\R^{n\times n}$ such that $AB=I_n$. 
                \pause

                We will show that $T$ is surjective by showing how we can get any $\vec{b}\in\R^{n}$.\pause\\
                Let $\vec{b}\in\R^{n}$. See that\pause
                \[
                    \vec{b} = I_n\vec{b}\pause = AB\vec{b}\pause = A(B\vec{b})\pause = T(B\vec{b})\pause
                \]
                Since we have an input, $\vec{x} = B\vec{b}$ such that $T(\vec{x}) = \vec{b}$,\pause

                We know that $T$ is surjective, so by above, $A$ is invertible.\pause

                Also, see that\pause
            \end{proof}
        }{}
    \end{frame}

    \begin{frame}{$BA=I_n$ case}
        \small
        \iftoggle{showSolutions}{
            \begin{proof}
                Let $B\in\R^{n\times n}$ such that $BA = I_n$.\pause\\
                We will show that the system $A\vec{x} = \vec{0}$ has only the trivial solution.\pause\\
                Assume that $\vec{x}\in\R^n$ such that $A\vec{x} = \vec{0}$.\pause\ See that\pause
                \begin{eqnarray*}
                    A\vec{x}    &=& \vec{0}\pause\\
                    BA\vec{x}   &=& B\vec{0}\pause\\
                    I_n\vec{x}  &=& \vec{0}\pause\\
                    \vec{x}     &=& \vec{0}\pause
                \end{eqnarray*}
                So, if $A\vec{x}=\vec{0}$, then $\vec{x}$ must be $\vec{0}!$\pause, which means $A$ 
                is invertible.\pause\\
                Next, we show $B=A^{-1}$\pause
                \[
                    A^{-1} = I_nA^{-1}\pause = BAA^{-1}\pause = BI_n\pause = B
                \]
            \end{proof}
        }{}
    \end{frame}
\end{document}
