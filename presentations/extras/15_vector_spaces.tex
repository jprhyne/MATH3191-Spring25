\documentclass[xcoler=dvipsnames, aspectratio=169]{beamer}

\usepackage{3191Style}
% Date gives the title of the lecture
\date{Vector Spaces}

\begin{document}
    \begin{frame}{What is a Vector Space?}
        \footnotesize
        A \rText{Vector Space} is a set $V$ that contains our \textit{vectors}, a set
        $F$ that contains our scalars with a \bText{vector addition} operation and 
        \bText{scalar multiplication} operation where the following properties are
        true for every $\vec{u},\vec{v},\vec{w}\in V$ and $c,d\in F$.
        \begin{columns}
            \column{.5\textwidth}
            \begin{enumerate}
                \pause\item $V$ is closed under addition:\pause\\
                    $\vec{u} + \vec{v}\in V$
                \pause\item Vector addition is commutative:\pause\\
                    $\vec{u}+\vec{v} = \vec{v}+\vec{u}$.
                \pause\item Vector addition is associative:\pause\\
                    $(\vec{u} + \vec{v}) + \vec{w} = \vec{u} + (\vec{v} + \vec{w})$
                \pause\item Additive Identity:\pause\\
                    There exists some $\vec{0}\in V$ where $\vec{v} + \vec{0} = \vec{v}$
                \pause\item Additive Inverse:\pause\ 
                    for each $\vec{u}\in V$, there is some $-\vec{u}\in V$ such that
                    $\vec{u} + (-\vec{u}) = \vec{0}$
            \end{enumerate}
            \column{.5\textwidth}
            \begin{enumerate}\addtocounter{enumi}{5}
                \pause\item $V$ is closed under scalar multiplication:\pause\\
                    $c\vec{v}\in V$.
                \pause\item Scalar multiplication distributes over vector addition:\pause\\
                    $c(\vec{u} + \vec{v}) = c\vec{u} + c\vec{v}$.
                \pause\item Scalar multiplication distributes over scalar addition:\pause\\
                    $(c+d)\vec{v} = c\vec{v} + d\vec{v}$
                \pause\item Scalar multiplication is associative:\pause\\
                    $c(d\vec{v}) = (cd)\vec{v}$
                \pause\item Multiplicative Identity:\pause\\
                    There exists some $1\in F$ such that $1\vec{u} = \vec{u}$.
            \end{enumerate}
        \end{columns}
    \end{frame}

    \begin{frame}{What are some examples?}
        $V=\R^n, F = \R$.\pause\ See Slide 8 of Lecture slide 3 for properties 2-5 and 7-10.

        For properties 1 and 2, we have the definitions of vector addition and scalar multiplication that guarantees this!
        \vspace{130pt}
    \end{frame}
    \begin{frame}{More Examples}
        $V = \P_2$ is the set of all polynomials with real coefficients of degree $2$ or less.\pause\\
        $F = \R$, with the operations we'd expect.
        \begin{columns}
            \column{.5\textwidth}
                \begin{enumerate}
                    \pause\item $(a_2x^2 + a_1x + a_0) + (b_2x^2 + b_1x + b_0) = 
                        (a_2 + b_2)x^2 + (a_1+b_1)x + (a_0 + b_0)$
                    \pause\item $f(x) + g(x) = g(x) + f(x)$
                    \pause\item $(f(x) + g(x)) + h(x) = f(x) + (g(x) + h(x))$
                    \pause\item $\vec{0} = 0x^2 + 0x + 0$
                    \pause\item $-f(x) = -a_2x^2 - a_1x - a_0$
                \end{enumerate}
            \column{.5\textwidth}
                \begin{enumerate}\addtocounter{enumi}{5}
                    \pause\item $cf(x) = ca_2x^2 + ca_1x + ca_0$
                    \pause\item $c(f(x) + g(x)) = cf(x) + cg(x)$
                    \pause\item $(c+d)f(x) = cf(x) + df(x)$
                    \pause\item $c(df(x)) = (cd)f(x)$
                    \pause\item $1f(x) = f(x)$
                \end{enumerate}
        \end{columns}
    \end{frame}
    \begin{frame}{Is this a Vector Space?}
        $V = \R^3, F=\R$ using standard scalar multiplication but $\vec{u}+\vec{v} =\begin{bmatrix}
            u_1+v_2\\
            u_2+v_2\\
            u_3+v_3
        \end{bmatrix}$
        \vspace{130pt}
    \end{frame}
    \begin{frame}{Uniqueness of $\vec{0}$}
        \begin{theorem} 
            If $V$ is a vector space, then the $\vec{0}$ element is unique
        \end{theorem}
        \iftoggle{showSolutions}{
            \pause
            \begin{proof}
                Let $\vec{w}\in V$ such that for every $\vec{u}\in V$ we have
                \[
                    \vec{w}+\vec{u} = \vec{u}+\vec{w} = \vec{u} 
                \]\pause
                By taking $\vec{u}=\vec{0}$, we have:\pause
                \begin{eqnarray*}
                    \vec{0}+\vec{w} &=& \vec{0}\pause\\
                    \vec{w}+\vec{0} &=& \vec{w}\pause
                \end{eqnarray*}
                Thus, we see that $\vec{w} = \vec{0}$
            \end{proof}
        }{\vspace{130pt}}
    \end{frame}
    \begin{frame}{Uniqueness of Additive Inverse}
        \begin{theorem}
            If $V$ is a vector space, then for every $\vec{u}\in V$, we have that
            $-\vec{u}$ is unique.
        \end{theorem}
        \iftoggle{showSolutions}{
            \pause
            \begin{proof}
                Let $\vec{u}\in V$, and suppose there are two additive identities IE that
                $-\vec{u}_1,-\vec{u}_2\in V$ such that $\vec{u}+(-\vec{u}_1) = \vec{0} =
                \vec{u} + (-\vec{u}_2)$ See that\pause
                \[
                    \vec{u} + (-\vec{u}_1) = 0\rightarrow -\vec{u}_2 + (\vec{u} + (-\vec{u}_1)) = -\vec{u}_2\pause\rightarrow (-\vec{u}_2 + \vec{u})+ -\vec{u}_1 = -\vec{u}_2
                \]
                \[
                    \pause\rightarrow -\vec{u}_1 = -\vec{u}_2
                \]
            \end{proof}
        }{\vspace{130pt}}
    \end{frame}
    \begin{frame}{Vector Space Practice}
        Work with your neighbors to determine if the following spaces are vector spaces
        \begin{columns}
            \column{.5\textwidth}
            $V = \R^3,F=\R$ with the usual vector addition and  $c\vec{u} = \begin{bmatrix}
                -cu_1\\-cu_2\\-cu_3
            \end{bmatrix}$.
            \column{.5\textwidth}
            $V = \R^{3\times 3},F=\R$  with the standard operations.
        \end{columns}
        \vspace{130pt}
    \end{frame}
    \begin{frame}{Vector Subspaces}
        \begin{defn}
            A subspace of a vector space $V$ is a subset $H$ of $V$ ($H\subseteq V$) that has the following
            properties (using the same vector addition, scalar multiplication, and $F$)
            \begin{enumerate}
                \pause\item The $\vec{0}$ from $V$ is in $H$.
                \pause\item $H$ is closed under vector addition: for each $\vec{u},\vec{v}\in H$, we have
                    $\vec{u} + \vec{v}\in H$.
                \pause\item $H$ is closed under scalar multiplication: for each $c\in F$
                    and $\vec{v}\in H$, we have $c\vec{v}\in  H$.
            \end{enumerate}
        \end{defn}
    \end{frame}
    \begin{frame}{Is it a Subspace?}
        Determine with your neighbors if each of the following sets are subspaces of
        $V=\R^3$.
        \begin{columns}
            \column{.5\textwidth}
            \[
                H = \set{\vec{v}\in\R^3}{\vec{v} = \begin{bmatrix}
                    3a+b\\
                    a+5\\
                    2a-5b
                \end{bmatrix}\textnormal{ for }a,b\in\R}
            \]
            \column{.5\textwidth}
            \[
                H = \set{\vec{v}\in\R^3}{\vec{v} = \begin{bmatrix}
                    a\\0\\b
                \end{bmatrix}\textnormal{ for }a,b\in\R}
            \]
        \end{columns}
        \vspace{130pt}
    \end{frame}
    \begin{frame}{Spanning Sets and Subspaces}
        Let $\vec{v}_1,\dots,\vec{v}_p$ denote a set of $p$ vectors in $V$. Then
        $\Span{\vec{v}_1,\dots,\vec{v}_p}$ is a subspace of $V$.
        \begin{enumerate}
            \pause\item $\Span{\vec{v}_1,\dots,\vec{v}_p}$ is the \rText{subspace spanned by $\vec{v}_1,\dots,\vec{v}_p$}
            \pause\item Given any subspace $H$ of $V$, a \rText{spanning set for $H$} is a set
                $\setBasic{\vec{v}_1,\dots,\vec{v}_p}$ of vectors in $H$ such that
                $H = \Span{\vec{v}_1,\dots,\vec{v}_p}$
        \end{enumerate}
    \end{frame}
    \begin{frame}{Example}
        Determine if $H=\set{A\in\R^{3\times 3}}{A = \begin{bmatrix}
            a & 0 & 0\\
            0 & b & 0\\
            0 & 0 & c
        \end{bmatrix}, a + b + c = 0}$ is a subspace of $\R^{3\times 3}$ and if so, give a spanning set
        for $H$.\pause
        Show that $\vec{0}\in H$, Show that we are closed under ``vector'' addition, and show we 
        are closed under scalar multiplication.
        \begin{enumerate}
            \pause\item Set $a=b=c=0$, clearly $a+b+c=0$ and then we have the $0$ matrix!
            \pause\item Let $A,B\in H$. See that
                \[
                    A + B = \begin{bmatrix}
                        a_1 + a_2 & 0 & 0\\
                        0 & b_1 + b_2 & 0\\
                        0 & 0 & c_1 + c_2
                    \end{bmatrix}
                \]\pause
                and $(a_1+a_2) + (b_1+b_2) + (c_1+c_2) = a_1 + b_1 + c_1 + a_2 + b_2 + c_2 \pause = 0 + 0 = 0$, so $A+B\in H$.
        \end{enumerate}
    \end{frame}
    \begin{frame}{Example continued}
        Determine if $H=\set{A\in\R^{3\times 3}}{A = \begin{bmatrix}
            a & 0 & 0\\
            0 & b & 0\\
            0 & 0 & c
        \end{bmatrix}, a + b + c = 0}$ is a subspace of $\R^{3\times 3}$ and if so, give a spanning set
        for $H$.
        \begin{enumerate}\addtocounter{enumi}{2}
            \pause\item let $x\in\R$. See that
                \[
                    xA= \begin{bmatrix}
                        xa & 0 & 0\\
                        0 & xb & 0\\
                        0 & 0 & xc
                    \end{bmatrix}
                \]\pause
                And 
                \[
                    xa + xb + xc = x(a+b+c)\pause = x\cdot 0 = 0\pause
                \]
        \end{enumerate}
        So, $H$ is a subspace of $\R^3$!
    \end{frame}
    \begin{frame}{Example continued pt. 2}
        See that our ``vectors'' are $3\times 3$ matrices, so our spanning set will 
        have these kinds of matrices!\pause\ 
        Define
        \[
            \vec{v}_1 = \begin{bmatrix}
                2 & 0 & 0\\
                0 &-1 & 0\\
                0 & 0 &-1
            \end{bmatrix}, \vec{v}_2 = \begin{bmatrix}
               -1 & 0 & 0\\
                0 & 2 & 0\\
                0 & 0 &-1
            \end{bmatrix}, \vec{v}_3 = \begin{bmatrix}
               -1 & 0 & 0\\
                0 &-1 & 0\\
                0 & 0 & 2
            \end{bmatrix}\pause
        \]
        See that
        \[
            \Span{\vec{v}_1,\vec{v}_2,\vec{v}_3} = \set{A\in\R^{3\times 3}}{
                A = a\vec{v}_1 + b\vec{v}_2 + c\vec{v}_3, \textnormal{ for }a,b,c\in\R
            }
        \]\pause
        And
    \end{frame}
    \begin{frame}{Example continued pt. 3}
        \[
            a\vec{v}_1 + b\vec{v}_2 + c\vec{v}_3 = a\begin{bmatrix}
                2 & 0 & 0\\
                0 &-1 & 0\\
                0 & 0 &-1
            \end{bmatrix} + b\begin{bmatrix}
               -1 & 0 & 0\\
                0 & 2 & 0\\
                0 & 0 &-1
            \end{bmatrix} + c\begin{bmatrix}
               -1 & 0 & 0\\
                0 &-1 & 0\\
                0 & 0 & 2
            \end{bmatrix}
        \]\pause 
        \[ = \begin{bmatrix}
                2a-b-c & 0 & 0\\
                0 & -a+2b-c & 0\\
                0 & 0 & -a-b+2c
            \end{bmatrix}
        \]
        Where
        \[
            2a-b-c+(-a+2b-c) + (-a-b+2c)\pause = 0
        \]
    \end{frame}
    \begin{frame}{Basis of a Vector Space}
        \begin{defn}
            \rText{Basis}: A \bText{basis} of a vector space $V$ is a set of $v_1,\dots,v_p\in V$
            such that
            \begin{enumerate}
                \pause\item $\Span{v_1,\dots,v_p} = V$ 
                \pause\item $v_1,\dots,v_p$ are linearly independent
            \end{enumerate}
        \end{defn}
        \pause
        \begin{example}
            \[
            \vec{v}_1 = \begin{bmatrix}
                2 & 0 & 0\\
                0 &-1 & 0\\
                0 & 0 &-1
            \end{bmatrix}, \vec{v}_2 = \begin{bmatrix}
               -1 & 0 & 0\\
                0 & 2 & 0\\
                0 & 0 &-1
            \end{bmatrix}, \vec{v}_3 = \begin{bmatrix}
               -1 & 0 & 0\\
                0 &-1 & 0\\
                0 & 0 & 2
            \end{bmatrix}\pause
            \]
            is a basis of
            \[
                H = \set{A\in\R^{3\times 3}}{A = \begin{bmatrix}
                    a & 0 & 0\\
                    0 & b & 0\\
                    0 & 0 & c
                \end{bmatrix}, a+b+c=0}
            \]
        \end{example}
    \end{frame}
    \begin{frame}{Length of Basis and Dimension of Vector Space}
        \begin{theorem}
            All bases of a vector space $V$ have the same number of elements
        \end{theorem}
        \vspace{100pt}
        \pause
        \begin{defn}
            The \bText{dimension} of a vector space, denoted $\dim{V}$ is the length of a basis
            of $V$.
        \end{defn}
    \end{frame}
    \begin{frame}{Spanning and Independent List of Correct Size is a Basis}
        \begin{theorem}
            Let $V$ be a vector space with $n=\dim{V}$. Then, any linearly independent list of 
            $n$ vectors, $\vec{v}_1,\dots,\vec{v}_n\in V$ forms a basis of $V$.
        \end{theorem}\pause\vspace{90pt}
        \begin{theorem}
            Let $V$ be a vector space with $n=\dim{V}$. Then, any spanning of 
            $n$ vectors, $\vec{v}_1,\dots,\vec{v}_n\in V$ such that $\Span{\vec{v}_1,\dots,\vec{v}_n} = V$
            is also a basis of $V$.
        \end{theorem}
        \vspace{130pt}
    \end{frame}
\end{document}
