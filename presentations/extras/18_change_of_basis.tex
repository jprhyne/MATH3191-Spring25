\documentclass[xcoler=dvipsnames, aspectratio=169]{beamer}

\usepackage{3191Style}
% Date gives the title of the lecture
\date{Change of Basis}
\newcommand{\B}{\mathcal{B}}
\newcommand{\C}{\mathcal{C}}
\begin{document}
    \begin{frame}{Change of Coordinate Matrix in $\R^n$}
        Recall that if $\B=\setBasic{\vec{b}_1,\dots,\vec{b}_n}$ is a basis of a vector space $V$, then
        the matrix
        \[
            P_\B = \bMat{\bCoord{\vec{b}_1}{\B} & \dots & \bCoord{\vec{b}_n}{\B}}
        \]
        Represents our basis vectors as an array of real numbers.\pause\
        
        What does this mean for $V=\R^n$?\pause\ Let's compute $P_\B\vec{x}$!\pause\ 
        \[
            P_\B\vec{x} = x_1\bCoord{\vec{b}_1}{\B} + \dots + x_n\bCoord{\vec{b}_n}{\B}
        \]\pause
        This means if $\vec{x}$ is an array of coordinates for the $\B$ basis, then the output is a
        vector written in the standard basis!
    \end{frame}
    \begin{frame}{Change of Coordinate Matrix in $\R^n$ Example}
        Consider the basis 
        \[
            \B = \setBasic{\bMat{1\\2\\1}, \bMat{0\\1\\2}, \bMat{2\\1\\0}}
        \]\pause\
        Then, $P_\B = \bMat{
            1 & 0 & 2\\
            2 & 1 & 1\\
            1 & 2 & 0
        }$. So, if we say that $\bCoord{\vec{x}}{\B} = \bMat{2\\1\\4}$, and we want to find what
        it is in the standard basis we compute
        \[
            \vec{x} = P_\B\bCoord{\vec{x}}{\B}\pause\ = \bMat{
            1 & 0 & 2\\
            2 & 1 & 1\\
            1 & 2 & 0
        }\bMat{2\\1\\4}\pause\ = 2\bMat{1\\2\\1} + 1\bMat{0\\1\\2} + 4\bMat{2\\1\\0}\pause\ =
        \bMat{10\\9\\4}
        \]
    \end{frame}
    \begin{frame}{Change of Coordinate Matrix in $\R^n$ is a Linear Transformation}
        \begin{theorem}
            If $V=\R^n$, then the Coordinate Matrix $P_\B$ is a Linear Transformation given by
            \[
                P_\B:\bCoord{\vec{x}}{\B}\mapsto\vec{x},
            \]
            which changes a vector from $\B$ coordinates to being written in the standard basis\pause\ 
            and 
            \[
                P^{-1}_\B:\vec{x}\mapsto\bCoord{\vec{x}}{\B}
            \]
            changes a vector from the standard basis back to $\B$ coordinates.
        \end{theorem}
    \end{frame}
    \begin{frame}{Changing Coordinate From One Basis to Another}
        In some data science applications, we may want to change problems from one basis to another.
        \pause
        \begin{theorem}
            Let $\B=\setBasic{\vec{b}_1, \dots, \vec{b}_n}$ and $\C=\setBasic{\vec{c}_1,\dots,\vec{c}_n}$
            be bases for a vector space $V$.\pause\ Then, there exists a unique matrix 
            $\underset{\C\leftarrow\B}{P}=P^{-1}_\C P_\B$
        \end{theorem}\pause
        
        This is really just saying that we (1) change our vector from being written in $\B$ coordinates
        to being written in the standard basis\pause\ , then (2) change our vector from the standard
        basis to $\C$ coordinates.\pause\ 

        Is there a potentially better way?
    \end{frame}
    \begin{frame}{A Better Way Without Explicitly Computing an Inverse}
        Yes!\pause\ 

        We also have that 
        \[
            \underset{\C\leftarrow\B}{P}=\bMat{\bCoord{\vec{b}_1}{\C} & \dots &\bCoord{\vec{b}_n}{\C}}
        \]\pause
        Which we can solve in one of two ways
        \begin{columns}
            \column{.5\textwidth}
                1) Solve
                \[
                    \aMat{c|c}{
                        P_\C & P_\B
                    }
                \]\pause
                Which is solving the equation
                \[
                    P_\C X = P_\B
                \]\pause
            \column{.5\textwidth}
                2) Solve
                \[
                    \aMat{c|c}{
                        P_\C & \vec{y}
                    }
                \]\pause
                Which is solving the equation
                \[
                    P_\C\vec{x} = \vec{y}
                \]
                then plugging in $\vec{y} = \vec{b}_1,\dots,\vec{b}_n$.\pause
        \end{columns}
        Either will give the right answer, it's just a matter of preference!
    \end{frame}
    \begin{frame}{Change of Coordinate Example}
        \footnotesize
        Consider the following bases of $\R^3$.
        \[
            \B = \setBasic{\bMat{1\\2\\1}, \bMat{0\\1\\2}, \bMat{2\\1\\0}}, \C = \setBasic{
                \bMat{1\\1\\1}, \bMat{0\\1\\1}, \bMat{0\\0\\1}}
        \]
        Compute $\underset{\C\leftarrow\B}{P}$.\pause
        First, we try method $1$.\pause
        \[
            \aMat{c|c}{
                P_\C & P_\B
            }\pause = \aMat{ccc|ccc}{
                1 & 0 & 0 & 1 & 0 & 2\\
                1 & 1 & 0 & 2 & 1 & 1\\
                1 & 1 & 1 & 1 & 2 & 0
            }\pause\rightarrow\aMat{ccc|ccc}{
                1 & 0 & 0 & 1 & 0 & 2\\
                0 & 1 & 0 & 1 & 1 & -1\\
                0 & 1 & 1 & 0 & 2 & -2
            }\pause\rightarrow\aMat{ccc|ccc}{
                1 & 0 & 0 & 1 & 0 & 2  \\
                0 & 1 & 0 & 1 & 1 & -1 \\
                0 & 0 & 1 & -1 & 1 & -1
            }
        \]
        So
        \[
            \underset{\C\leftarrow\B}{P} = \bMat{
                1 & 0 & 2\\
                1 & 1 & -1\\
                -1 & 1 & -1
            }
        \]
    \end{frame}
    \begin{frame}{Change of Coordinate Example}
        \footnotesize
        Consider the following bases of $\R^3$.
        \[
            \B = \setBasic{\bMat{1\\2\\1}, \bMat{0\\1\\2}, \bMat{2\\1\\0}}, \C = \setBasic{
                \bMat{1\\1\\1}, \bMat{0\\1\\1}, \bMat{0\\0\\1}}
        \]
        Compute $\underset{\C\leftarrow\B}{P}$.
        Next, we try method $2$.\pause
        \[
            \aMat{c|c}{
                P_\C & \vec{y}
            }\pause = \aMat{ccc|c}{
                1 & 0 & 0 & y_1\\
                1 & 1 & 0 & y_2\\
                1 & 1 & 1 & y_3
            }\pause\rightarrow\aMat{ccc|c}{
                1 & 0 & 0 & y_1\\
                0 & 1 & 0 & y_2-y_1\\
                0 & 1 & 1 & y_3-y_1
            }\pause\rightarrow\aMat{ccc|c}{
                1 & 0 & 0 & y_1\\
                0 & 1 & 0 & y_2-y_1\\
                0 & 0 & 1 & y_3-y_2
            }
        \]
        So the columns of $\underset{\C\leftarrow\B}{P}$ are given by
        \[
            \bCoord{\vec{b}_1}{\C} = \bMat{1\\1\\-1}, \bCoord{\vec{b}_2}{\C} = \bMat{0\\1\\1}, 
            \bCoord{\vec{b}_3}{\C} = \bMat{2\\-1\\-1}\pause\rightarrow\underset{\C\leftarrow\B}{P} = \bMat{
                1 & 0 & 2\\
                1 & 1 & -1\\
                -1 & 1 & -1
            }
        \]
    \end{frame}
    \begin{frame}{Change of Coordinate Practice}
        Consider the following bases of $\R^3$.
        \[
            \B = \setBasic{\bMat{1\\1\\1}, \bMat{0\\1\\1}, \bMat{0\\0\\1}},
            \C = \setBasic{\bMat{1\\2\\1}, \bMat{0\\1\\2}, \bMat{2\\1\\0}} 
        \]\pause
        Compute $\underset{\C\leftarrow\B}{P}$
        \iftoggle{showSolutions}{
            \pause
            \[
                \underset{\C\leftarrow\B}{P} = \bMat{
                    0 & \frac{1}{2} & -\frac{1}{2}\\
                    \frac{1}{2} & \frac{1}{4} & \frac{3}{4}\\
                    \frac{1}{2} & -\frac{1}{4}& \frac{1}{4}
                }
            \]
        }{
        }
        \vspace{130pt}
    \end{frame}
    \begin{frame}{Matrix Vector Space Practice}
        Let $V$ be the vector space of \rText{symmetric $2\times 2$} real matrices. We have the
        basis
        \[
            \B=\setBasic{\bMat{1&0\\0&0}, \bMat{0&1\\1&0}, \bMat{0&0\\0&1}}.
        \]
        Consider the set
        $
            \C = \setBasic{\bMat{1&-2\\-2&2}, \bMat{0&2\\2&4}, \bMat{3&-1\\-1&1}}.
        $
        Answer the following questions
        \begin{columns}
            \column{.5\textwidth}
            \bText{1)} Show that $\C$ is a basis for $V$.

            \iftoggle{showSolutions}{
                \only<2->{
                    Set up and show 
                    \[
                        \underset{\B\leftarrow\C}{P} = \bMat{
                            1 & 0 & 3\\
                            -2 & 2 & -1\\
                            2 & 4 & 1
                        }
                    \]
                    is invertible.
                }
            }{\vspace{130pt}}
            \column{.5\textwidth}
            \bText{2)} Write the matrix corresponding to
            $
                \bCoord{\bMat{2\\3\\-2}}{\C}
            $ in $\B$ coordinates

            \iftoggle{showSolutions}{
                \only<3->{
                    \[
                        \bMat{
                            -4 & 4\\
                            4 & 14
                        }
                    \]
                }
            }{\vspace{100pt}}
        \end{columns}
    \end{frame}
\end{document}
