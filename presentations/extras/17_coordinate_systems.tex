\documentclass[xcoler=dvipsnames, aspectratio=169]{beamer}

\usepackage{3191Style}
% Date gives the title of the lecture
\date{Coordinate Systems}
\newcommand{\B}{\mathcal{B}}
\begin{document}
    \begin{frame}{Uniqueness Representation Theorem}
        \begin{theorem}
            Let $\B=\setBasic{\vec{v}_1,\vec{v}_2,\dots,\vec{v}_n}$ be a basis for a vector space $V$.\pause
            Then, for every $\vec{v}\in V$, there is a unique set of $c_1,\dots,c_n$ such that
            \[
                \vec{v} = c_1\vec{v}_1 + \dots + c_n\vec{v}_n
            \]
        \end{theorem}
    \end{frame}
    \begin{frame}{Proof of Uniqueness Representation Theorem}
        \begin{proof}
            Since $\vec{v}_1,\dots,\vec{v}_n$ forms a basis of $V$, then we know that 
            \begin{enumerate}
                \pause\item $\Span{\vec{v}_1,\dots,\vec{v}_n}=V$
                \pause\item $\vec{v}_1,\dots,\vec{v}_n$ are all linearly independent.
            \end{enumerate}\pause\ 
            Define $A = \bMat{\vec{v}_1 & \dots & \vec{v}_n}$, then since the columns of $A$ span
            all of $V$, we know that we can solve $A\vec{c} = \vec{v}$ for every $\vec{v}\in V$.\pause\ 
            Since the columns of $A$ are linearly independent, we know $\vec{c}$ is unique. \pause
            Putting this all together means
            \[
                \vec{v} = A\vec{c} = c_1\vec{v}_1 + \dots + c_n\vec{v}_n.
            \]
        \end{proof}
    \end{frame}
    \begin{frame}{Basis as a Coordinate System}
        \begin{defn}
            Let $\B=\setBasic{\vec{v}_1,\dots,\vec{v}_n}$ be a basis for a vector space $V$. Then,
            for every $\vec{x}\in V$, we define the \bText{$\B$ coordinates of $\vec{x}$} to be 
            the unique scalars $c_1,\dots,c_n$ such that $\vec{x} = c_1\vec{v}_1 + \dots + c_n\vec{v}_n$.
        \end{defn}
        \vspace{45pt}
        \begin{defn}
            We define the \rText{$\B$-coordinate vector of $\vec{x}$} to be:
            \[
                \bCoord{\vec{x}}{\B} = \bMat{c_1\\\vdots\\c_n}
            \]
        \end{defn}
    \end{frame}
    \begin{frame}{Coordinate System Example}
        Let $V=\R^{2\times 2}$ and consider the basis given by
        \[
            \B = \setBasic{\bMat{1 & 0\\0 & 0}, \bMat{0 & 1\\0 & 0}, \bMat{0 & 0\\1 & 0}, \bMat{0 & 0\\
            0 & 1}}
        \]
        And label them as follows
        \[
            \vec{v}_1 = \bMat{1 & 0\\0 & 0},\vec{v}_2 = \bMat{0 & 1\\0 & 0}, 
            \vec{v}_3 =\bMat{0 & 0\\1 & 0} ,\vec{v}_4 = \bMat{0 & 0\\ 0 & 1}
        \]
        Then 
        \[
            \bCoord{\bMat{1 & 5\\-4 & 15}}{\B} = \bMat{1\\5\\-4\\15}
        \]
    \end{frame}
    \begin{frame}{Coordinate System Example pt. 2}
        \footnotesize
        Let $V=\R^3$, $F=\R$, and $\B=\setBasic{\bMat{1\\-1\\1}, \bMat{0\\3\\1}, \bMat{2\\-1\\0}}$
        be a basis of $V$.\pause\ Find
        \[
            \bCoord{\bMat{1\\0\\1}}{\B}
        \]\pause\
        We want to solve the following augmented system:\pause\
        \[
            \aMat{ccc|c}{
                1 & 0 & 2 & 1\\
                -1& 3 &-1 & 0\\
                1 & 1 & 0 & 1
            }\pause\xrightarrow[R_3=R_3-R_1]{R_2=R_2+R_1}\aMat{ccc|c}{
                1 & 0 & 2 & 1\\
                0 & 3 & 1 & 1\\
                0 & 1 &-2 & 0
            }\pause\xrightarrow{R_2\leftrightarrow R_3}\aMat{ccc|c}{
                1 & 0 & 2 & 1\\
                0 & 1 &-2 & 0\\
                0 & 3 & 1 & 1
            }\pause\xrightarrow{R_3=R_3-3R_1}\aMat{ccc|c}{
                1 & 0 & 2 & 1\\
                0 & 1 &-2 & 0\\
                0 & 0 & 7 & 1
            }
        \]
        \[
            \pause\xrightarrow{R_3=\frac{1}{7}R_3}\aMat{ccc|c}{
                1 & 0 & 2 & 1\\
                0 & 1 &-2 & 0\\
                0 & 0 & 1 & \frac{1}{7}
            }\pause\xrightarrow[R_1=R_1-2R_3]{R_2=R_2+2R_3}\aMat{ccc|c}{
                1 & 0 & 0 & -\frac{2}{7}\\
                0 & 1 & 0 & \frac{2}{7}\\
                0 & 0 & 1 & \frac{1}{7}
            }\pause\rightarrow 
            \bCoord{\bMat{1\\0\\1}}{\B} = \bMat{-\frac{2}{7}\\\frac{2}{7}\\\frac{1}{7}}
        \]
    \end{frame}
    \begin{frame}{Coordinate System Practice}
        Let $V=\R^3$, $F=\R$, and $\B=\setBasic{\bMat{1\\0\\1}, \bMat{2\\1\\1}, \bMat{0\\0\\-1}}$
        be a basis of $V$.\pause\ 
        Find 
        \[
            \bCoord{\bMat{1\\1\\3}}{\B}
        \]
        \iftoggle{showSolutions}{
            \pause
        \[
            \bCoord{\bMat{1\\1\\3}}{\B} = \bMat{-1\\1\\-3}
        \]
        }{
            \vspace{100pt}
        }
    \end{frame}
    \begin{frame}{Coordinate Mapping}
        \begin{defn}
            \rText{Coordinate Mapping}: The mapping $C: \vec{x}\mapsto\bCoord{\vec{x}}{\B}$
            where $\B$ is some basis of $V$, $n=\dim{\B}$, and we call $C$ the \bText{coordinate mapping}.
        \end{defn}
        \pause
        Another way to phrase this is as follows:\pause

        \vspace{10pt}
        Let $\B=\setBasic{\vec{v}_1,\dots,\vec{v}_n}$, and define $A=\bMat{\vec{v}_1 & \dots & \vec{v}_n}$
        to be a matrix whose columns are the basis vectors from $\B$.\pause

        \vspace{10pt}
        \emph{Note}: This may not be a matrix of the usual real numbers, could be functions, matrices, etc!\pause

        Then for all $\vec{v}\in V$, define $\vec{x} = C(\vec{v})$, then $\vec{v} = A\vec{x}$.\pause
        \vspace{10pt}

        We will now prove some properties of $C$!
    \end{frame}
    \begin{frame}{Coordinate Mapping is an Injection}
        Remember that $A = \bMat{\vec{v}_1,\dots,\vec{v}_n}$. We will show that the coordinate mapping
        is an injection first.\pause
        \begin{proof}
            Let $\vec{u},\vec{v}$ be two vectors in $V$ such that $C(\vec{u}) = C(\vec{v})$.\pause
            
            Since $C(\vec{u}) = C(\vec{v})$, there exists some $\vec{x}\in\R^n$ such that
            \[
                A\vec{x} = \vec{u}\qquad\qquad A\vec{x}=\vec{v}
            \]\pause
            Since $A$ is a matrix whose columns are vectors, we can see that
            \[
                \vec{u}-\vec{v} = A\vec{x} - A\vec{x}\pause = A(\vec{x}-\vec{x})\pause = 
                A(\vec{0}_n)\pause = \vec{0}_V
            \]\pause
            So, $\vec{u}=\vec{v}$.
        \end{proof}
    \end{frame}
    \begin{frame}{Coordinate Mapping is a Surjection}
        Remember that $A = \bMat{\vec{v}_1,\dots,\vec{v}_n}$. We will show that the coordinate mapping
        is a surjection.\pause\ 
        We want to show that we can get every $\vec{x}\in\R^n$ by choosing the correct $\vec{v}\in V$.\pause
        \begin{proof}
            Let $\vec{x}\in\R^n$. See that $A\vec{x} = \vec{v}\in V$.\pause\ 
            Then by definition of $C$, we have that $C(\vec{v}) = \vec{x}$.\pause\ Thus, we can reach every element in $\R^n$.\pause

            Therefore, $C$ is a surjection
        \end{proof}
    \end{frame}
    \begin{frame}{Coordinate Mapping is a Bijection}
        \begin{theorem}
            The \bText{coordinate mapping} $C:V\rightarrow\R^n$ is a bijection.
        \end{theorem}\pause
        \begin{proof}
            The last 2 slides proved that $C$ is an injection and surjection. Therefore it is a 
            bijection.
        \end{proof}\pause
        \vspace{60pt}
        \begin{tcolorbox}
            We also call bijections \rText{isomorphisms}.\pause\ This just means we can think about
            the vectors and their coordinate vectors interchangeably!
        \end{tcolorbox}
    \end{frame}
    \begin{frame}{Coordinate Matrix}
        \small
        \begin{defn}
            \rText{Coordinate Matrix}: We define the \bText{coordinate matrix} of a list of
            vectors: $\vec{v}_1,\dots,\vec{v}_p$ in the basis $\B=\setBasic{\vec{b}_1,\dots,\vec{b}_n}$
            as follows:
            \[
                P_\B = \bMat{\bCoord{\vec{v}_1}{\B} & \dots & \bCoord{\vec{v}_p}{\B}}
            \]
        \end{defn}\pause
        \begin{example}
            Consider $V=\R^{2\times 2},F=\R$ with the basis 
            $\B=\setBasic{\bMat{1&0\\0&0},\bMat{0&1\\0&0},\bMat{0&0\\1&0},\bMat{0&0\\0&1}}$.\pause\ 
            The Coordinate matrix for $\bMat{1&4\\0&1},\bMat{2&4\\1&1},\bMat{0&0\\4&2}$ is given by:\pause\
            \[
                P_\B = \bMat{
                    1 & 2 & 0\\
                    4 & 4 & 0\\
                    0 & 1 & 4\\
                    1 & 1 & 2
                }
            \]
        \end{example}
    \end{frame}
    \begin{frame}{But Why do we Care?}
        \footnotesize
        In weird spaces, it's easier to check for pivot columns in $P_\B$ than it is to check if
        $\vec{v}_1,\dots,\vec{v}_p$ are linearly independent!\pause\
        Consider $V=\P_2(\R),F=\R$. (Note: $\P_2(\R)$ is the set of all polynomial of degree 2 or less
        with real coefficients.)\pause\

        Consider the ``standard'' basis of $\B=\setBasic{1,t,t^2}$. Let's check if 
        $1-t+t^2, 3t+t^2, 2-t$ are linearly independent!\pause\
        \[
            P_\B = \bMat{
                1 & 0 & 2\\
                -1& 3 &-1\\
                1 & 1 & 0
            }
        \]\pause
        \vspace{-5pt}
        \[
            \bMat{
                1 & 0 & 2\\
                -1& 3 &-1\\
                1 & 1 & 0
            }\pause\xrightarrow[R_3=R_3-R_1]{R_2=R_2+R_1}\bMat{
                1 & 0 & 2\\
                0 & 3 & 1\\
                0 & 1 &-2
            }\pause\xrightarrow{R_2\leftrightarrow R_3}\bMat{
                1 & 0 & 2\\
                0 & 1 &-2\\
                0 & 3 & 1
            }\pause\xrightarrow{R_3=R_3-3R_2}\bMat{
                1 & 0 & 2\\
                0 & 1 &-2\\
                0 & 0 & 7
            }
        \]
        \vspace{-5pt}
        \[
            \pause\xrightarrow{R_3=\frac{1}{7}R_3}\bMat{
                1 & 0 & 2\\
                0 & 1 &-2\\
                0 & 0 & 1
            }\pause\xrightarrow[R_1=R_1-2R_3]{R_2=R_2+2R_3}\bMat{
                1 & 0 & 0\\
                0 & 1 & 0\\
                0 & 0 & 1
            }\pause
        \]
        Thus, these vectors are linearly independent!
    \end{frame}
\end{document}
