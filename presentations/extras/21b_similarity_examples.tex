\documentclass[xcoler=dvipsnames, aspectratio=169]{beamer}

\usepackage{3191Style}
% Date gives the title of the lecture
\date{Similarity Practice}

\begin{document}
    \begin{frame}{Example 1 Part 1}
        Let 
        \[
            A = \bMat{
                4   & 14 & -4\\
                -12 & 28 & -8\\
                12  & -13& 18
            }, \qquad B=\bMat{
                10 & 0 & 0\\
                0 & 20 &10\\
                0 & 0 & 20
            }, \qquad C=\bMat{
                2 & 2 & -3\\
                0 & 2 & -2\\
                -3&-1 & 0
            }
        \]
        We will demonstrate that $A$ is similar to $B$ using $C$.\pause\ Compute $AC$, and $CB$.
        \[
            AC\pause = \bMat{
                4   & 14 & -4\\
                -12 & 28 & -8\\
                12  & -13& 18
            }\bMat{
                2 & 2 & -3\\
                0 & 2 & -2\\
                -3&-1 & 0
            } \pause = \bMat{
                20  & 40 & -40\\
                0   & 40 & -20\\
                -30 &-20 & -10
            }
        \]
    \end{frame}
    \begin{frame}{Example 1 Part 2}
        \[
            AC = \bMat{
                20  & 40 & -40\\
                0   & 40 & -20\\
                -30 &-20 & -10
            }
        \]\pause
        Now we compute $CB$
        \[
            CB\pause = \bMat{
                2 & 2 & -3\\
                0 & 2 & -2\\
                -3&-1 & 0
            }\bMat{
                10 & 0 & 0\\
                0 & 20 & 10\\
                0 & 0 & 20
            }\pause = \bMat{
                20  & 40 & -40\\
                0   & 40 & -20\\
                -30 &-20 & -10
            }
        \]\pause
        Which is the same as $AB$.
    \end{frame}

    \begin{frame}{Example 2 Part 1}
        \small
        Let $A,B$ be given below, then compute a value of $k$ such that $A$ and $B$ are similar.
        \[
            A = \bMat{
                5 & 2\\
                4 & k
            },\qquad B=\bMat{
                1 & 0\\
                0 & 7
            }
        \]\pause
        This is just asking us to find a $k$ such that $A$ has eigenvalues $1,7$ and is 
        diagonalizable!\pause\ There are many ways to approach this. The easiest is probably to 
        write out the characteristic polynomial we want
        \[
            f(\lambda) = (\lambda-1)(\lambda-7) \pause = \lambda^2 - 8\lambda + 7
        \]\pause
        Then finding the characteristic polynomial of $A$ and setting them equal
        \[
            \det{\bMat{
                5-\lambda & 2\\
                4 & k-\lambda
            }} = (5-\lambda)(k-\lambda) - 8 \pause = \lambda^2 - (5+k)\lambda + 5k-8
        \]
    \end{frame}
    \begin{frame}{Example 2 Part 2}
        So, we will now set these polynomials equal to each other\pause
        \[
            \lambda^2 - 8\lambda + 7 = \lambda^2 - (5+k)\lambda + 5k-8
        \]\pause
        Remember that $\lambda$ is our variable here, so we treat $k$ as a constant. We are left
        with the following system to solve
        \begin{eqnarray*}
            -(5+k) &=& -8\\
            5k-8   &=& 7
        \end{eqnarray*}\pause
        We can solve this however we want, but the first equation gives us $k=3$ and plugging this into
        the second gives us what we need, so $k$ must be $3$.
    \end{frame}
    \begin{frame}{Example 2 Part 3}
        This means that we showed
        \[
            \bMat{
                5 & 2\\
                4 & 3
            }\textnormal{, and }\bMat{
                1 & 0\\
                0 & 7
            }
        \]
        have the same eigenvalues.\pause\ Are we done?\pause\ \rText{No!}\pause\ Just because
        two matrices have the same eigenvalues doesn't mean that they are similar.\pause\ We need to compute
        the $C$ that proves these matrices are similar.\pause\

        Now we find the eigenvectors of $A$ and then we would be done!
    \end{frame}
\end{document}
