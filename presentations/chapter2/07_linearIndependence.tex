\documentclass[xcoler=dvipsnames, aspectratio=169]{beamer}
\usepackage{3191Style}
\date{Linear Independence}

\begin{document}
\begin{frame}{Revisiting Vector Equations and Homogeneous Systems}
    \begin{tcolorbox}
        If $A\in\R^{m\times n}$ with columns $\vec{a}_1,\dots,\vec{a}_n\in\R^m$, then the 
        homogeneous matrix equation $A\vec{x} = \vec{0}$ has the same solution set as the vector 
        equation
        \[
            x_1\vec{a}_1 + \dots + x_n\vec{a}_n = \vec{0}
        \]
        which has the corresponding augmented matrix
        \[
            \aMat{cccc|c}{
                \vec{a}_1 & \vec{a}_2 & \dots & \vec{a}_n & \vec{0}
            }
        \]
    \end{tcolorbox}
    \pause
    \begin{itemize}
        \item Homogeneous Linear Systems always have the trivial solution: $\vec{x}=\vec{0}$\pause
        \item If a non-trivial solution exists, then: \pause
            \begin{itemize}
                \item At least one $x_i\neq 0$\pause
                \item At least one $\vec{a}_i$ can be written as a linear combination of the others!\pause
            \end{itemize}
        \item If there is only the trivial solution, then no columns can be written as a linear combination
            of the others!
    \end{itemize}
\end{frame}
\begin{frame}{Brief Proof (on board) Of the last bullet point.}
\end{frame}
\begin{frame}[fragile]{Example}
    Determine whether the equation
    \[
        x_1\begin{bmatrix}2\\1\\0\end{bmatrix} + x_2\begin{bmatrix}-1\\6\\3\end{bmatrix} + x_3
            \begin{bmatrix}6\\-10\\-6\end{bmatrix} = \begin{bmatrix}0\\0\\0\end{bmatrix}
    \]
    has a non-trivial solution.
    \pause
    \begin{columns}
        \column{.5\textwidth}
        \[
            \aMat{ccc|c}{
                2 & -1 & 6 & 0\\
                1 & 6  &-10& 0\\
                0 & 3  &-6 & 0
            }\pause\rightarrow\aMat{ccc|c}{
                1 & 0 & \rTextWait{2}{4-} & 0 \\
                0 & 1 & \rTextWait{-2}{4-} & 0 \\
                0 & 0 & \rTextWait{0}{4-} & 0
            }
        \]\pause
        Since $\rText{x_3}$ is a free variable, we have non-trivial solutions
        \column{.5\textwidth}\pause
        Python code to get this answer!
        \begin{verbatim}
import sympy as sym
A = sym.Matrix([[2, -1, 6, 0],
                [1, 6, -10, 0],
                [0, 3, -6, 0]])
display(A.rref())
        \end{verbatim}
    \end{columns}
\end{frame}
\begin{frame}{What does it mean?}
    \begin{columns}
        \column{.5\textwidth}
        We get
        \begin{align*}
            x_1 + 2x_3 &= 0 \onslide<2->{\rightarrow x_1 = -2x_3}\\
            x_2 - 2x_3 &= 0 \onslide<2->{\rightarrow x_2 = 2x_3}
        \end{align*}
        \pause
        \pause
        Where the solution set is
        \[
            \vec{x} = x_3\begin{bmatrix}
                -2\\2\\1
            \end{bmatrix}
        \]\pause
        \column{.5\textwidth}
        Plugging in gives:
        \[
            \vec{0} = A\vec{x} = \rText{-2}\begin{bmatrix}2\\1\\0\end{bmatrix}
                + \rText{2}\begin{bmatrix}-1\\6\\3\end{bmatrix}
                + \rText{1}\begin{bmatrix}6\\-10\\-6\end{bmatrix}
        \]
        \pause
        Or equivalently
        \[
            \begin{bmatrix}6\\-10\\-6\end{bmatrix} = \rText{2}\begin{bmatrix}2\\1\\0\end{bmatrix}
                + \rText{-2}\begin{bmatrix}-1\\6\\3\end{bmatrix}
        \]
        \pause
        So, one is a linear combination of the others!
    \end{columns}
\end{frame}
\begin{frame}{Linear Independence}
    \begin{defn}
        \rText{Linear Independence}: We say that a set of $p>1$ vectors in $\R^n$, 
        $\left\{\vec{v}_1,\dots,\vec{v}_p\right\}$ is \bText{linearly independent} if we \textbf{cannot} 
        write one as a linear combination of the others. Otherwise, the set is \bText{linearly dependent}.
    \end{defn}
    \begin{example}
        Since 
        \[
            \begin{bmatrix}6\\-10\\-6\end{bmatrix} = \rText{2}\begin{bmatrix}2\\1\\0\end{bmatrix}
                + \rText{-2}\begin{bmatrix}-1\\6\\3\end{bmatrix},
        \]
        we know that the set $\left\{\begin{bmatrix}2\\1\\0\end{bmatrix},
            \begin{bmatrix}-1\\6\\3\end{bmatrix},\begin{bmatrix}6\\-10\\-6\end{bmatrix}\right\}$ is\pause\ 
                \bText{linearly dependent}!
    \end{example}
\end{frame}
\begin{frame}{Showing Linear Independence}
    \begin{theorem}
        To show $\left\{\vec{v}_1,\dots,\vec{v}_p\right\}$ is \rText{linearly independent}, we can
        just show that the system\pause
        \[
            \aMat{ccc|c}{
                \vec{v}_1 & \dots & \vec{v}_p & \vec{0}
            }
        \]
        \pause
        only has the trivial solution!
    \end{theorem}
    \vspace{60pt}\pause
    \begin{tcolorbox}
        If we find another solution, then they are \bText{linearly dependent}
    \end{tcolorbox}
\end{frame}
\begin{frame}{Practice!}
    Determine if the following sets of vectors are \bText{linearly dependent} or \bText{linearly
    independent}
    \begin{columns}
        \column{.5\textwidth}
        \begin{itemize}
            \item
                \[
                    S = \left\{\begin{bmatrix}1\\2\end{bmatrix},\begin{bmatrix}3\\6\end{bmatrix}\right\}
                \]
        \end{itemize}
        \onslide<2->{
            \iftoggle{showSolutions}{
                Since $\begin{bmatrix}3\\6\end{bmatrix} = 3\begin{bmatrix}1\\2\end{bmatrix}$, they are
                    \bText{linearly dependent}
            }{}
        }
        \column{.5\textwidth}
        \begin{itemize}
            \item 
                \[
                    S = \left\{\begin{bmatrix}1\\2\\1\end{bmatrix},\begin{bmatrix}3\\6\\1\end{bmatrix}\right\}
                \]
        \end{itemize}
        \onslide<3->{
            \iftoggle{showSolutions}{
                They are \bText{linearly independent}! Perform Gaussian Elimination on the Augmented matrix to see this
            }{}
        }
    \end{columns}
\end{frame}
\begin{frame}{Conceptual Practice!}
    For each of the following questions, either come up with an example or think of a reason it's false
    \begin{enumerate}
        \item Give an example of $1$ vector in $\R^3$ so that the set is linearly independent

            \onslide<2->{
            \iftoggle{showSolutions}{
                \rText{Any vector will do!}}{}
        }
        \item Give an example of $2$ vector in $\R^3$ so that the set is linearly independent

            \onslide<3->{
            \iftoggle{showSolutions}{
                \rText{$e_1$ and $e_2$ work!}}{}
        }
        \item Give an example of $3$ vector in $\R^3$ so that the set is linearly independent

            \onslide<4->{
            \iftoggle{showSolutions}{
                \rText{This is the last one we can do!}}{}
        }
        \item Give an example of $4$ vector in $\R^3$ so that the set is linearly independent

            \onslide<5->{
            \iftoggle{showSolutions}{
                \rText{This is not possible}}{}
        }
    \end{enumerate}
\end{frame}
\begin{frame}{But what about $\vec{0}$?}
    \begin{theorem}
        If a set $S=\left\{\vec{v}_1,\dots,\vec{v}_p\right\}$ contains the zero vector, $\vec{0}$,
        then the set is linearly dependent.
    \end{theorem}
    \vspace{60pt}
    \begin{proof}
        Since we can reorder the list without changing the overall property, let $\vec{v}_1=\vec{0}$.\pause

        See that
        \[
            1\vec{0} + 0\vec{v}_2 + \dots + 0\vec{v}_p = \vec{0}
        \]
    \end{proof}
\end{frame}
\begin{frame}{Number of Vectors and Their Dimension}
    \begin{theorem}
        A set of $p$ vectors, $S=\left\{\vec{v}_1,\dots,\vec{v}_p\right\}$ in $\R^n$ is linearly
        dependent if $p>n$.
    \end{theorem}
    \vspace{200pt}
\end{frame}
\end{document}
