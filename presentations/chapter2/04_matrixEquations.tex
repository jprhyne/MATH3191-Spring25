\documentclass[xcoler=dvipsnames, aspectratio=169]{beamer}
\usepackage{3191Style}
\date{Matrix Equations}

\begin{document}
    \begin{frame}
        \begin{definition}{Matrix-Vector Products}
            \rText{Matrix-Vector Product}: Let $A\in\R^{m\times n}$ be an $m\times n$ matrix\pause\ 
            where each column is given by $\vec{a}_1,\dots, \vec{a}_n$ are column vectors in $\R^m$\pause\pause , and let 
            $\vec{x}\in\R^n$ be a column vector. Then we define the \bText{matrix-vector product} of $A$ and $x$
            to be
            \pause
            \[
                \onslide<3->A\onslide<5->\vec{x} \onslide<3->= \begin{bmatrix}
                    \vec{a}_1 & \dots & \vec{a}_n
                \end{bmatrix}\onslide<5->\begin{bmatrix}
                    x_1\\\vdots\\x_n
                \end{bmatrix} \pause= \vec{a}_1x_1 + \cdots + \vec{a}_nx_n \pause= \sum_{k=1}^n \vec{a}_kx_k
            \]
        \end{definition}
        \pause
        \begin{tcolorbox}
            What does this mean about the relation between the columns and $A$ and the number of 
            elements in $\vec{x}$?\iftoggle{showSolutions}{
                \rText{$\vec{x}$ must have the same number of elements as $A$ does rows!}
            }{}
        \end{tcolorbox}
    \end{frame}
    \begin{frame}{Matrix-Vector Product Practice}
        \small
        Determine if each of the following matrix-vector products are defined. If they are, then
        compute the product.
        \begin{columns}
            \column{.5\textwidth}
            \begin{enumerate}[a)]
                \item $\displaystyle A = \onslide<2>\underbrace{\onslide<1->\begin{bmatrix}
                        2 & -4 & 3 & 1\\
                        6 & 2 & 1 & 9 \\
                        1 & 0 & 2 & -1\\
                \end{bmatrix}\onslide<2>}_{\rText{4}}\onslide<1->$ and $\displaystyle \vec{x} = 
                    \onslide<2>\left.\onslide<1->\begin{bmatrix}
                    -1\\-1\\5\\2
                    \end{bmatrix}\onslide<2>\right\}\bText{4}\onslide<1->$
            \end{enumerate}
            \column{.5\textwidth}
            \begin{enumerate}[a)]
                    \addtocounter{enumi}{1}
                \item $\displaystyle A = \onslide<3>\underbrace{\onslide<1->\begin{bmatrix}
                        2 & -4 & 3 & 1\\
                        6 & 2 & 1 & 9 \\
                        1 & 0 & 2 & -1\\
                \end{bmatrix}\onslide<3> }_{\rText{4}}\onslide<1->$ and $\displaystyle \vec{x} = 
                    \onslide<3->\left.\onslide<1->\begin{bmatrix}
                    -1\\-1\\5
                \end{bmatrix}\onslide<3>\right\}\bText{3}\onslide<1->$
            \end{enumerate}
        \end{columns}
        \vfill
        \pause\pause\pause
        For (a):

        \[
            A\vec{x} = \pause\begin{bmatrix}
                        2 & -4 & 3 & 1\\
                        6 & 2 & 1 & 9 \\
                        1 & 0 & 2 & -1\\
                \end{bmatrix}\begin{bmatrix}
                    -1\\-1\\5\\2
                \end{bmatrix}\pause = -1\begin{bmatrix}2\\6\\1\end{bmatrix}-1\begin{bmatrix}-4\\2\\0\end{bmatrix}
                    +5\begin{bmatrix}3\\1\\2\end{bmatrix} + 2\begin{bmatrix} 1\\9\\-1\end{bmatrix}
                       \pause = \begin{bmatrix}
                            -2+4+15+2\\
                            -6-2+5+18\\
                            -1-0+10-2
                        \end{bmatrix} \pause= \begin{bmatrix}
                            19\\15\\7
                        \end{bmatrix}
        \]
    \end{frame}
        \begin{frame}{Matrix-Vector Product Properties}
            \scriptsize
            \begin{tcolorbox}
                Let $A\in\R^{m\times n}$, $\vec{u},\vec{v}\in\R^n$, and $c$ be a scalar. Then we know
                \begin{itemize}
                    \item $A(\vec{u}+\vec{v}) = A\vec{u} + A\vec{v}$
                    \item $A(c\vec{v}) = cA\vec{v}$
                \end{itemize}
            \end{tcolorbox}
                \vfill
                \pause
                \begin{proof}
                    We will demonstrate that $A(\vec{u}+\vec{v}) = A\vec{u} + A\vec{v}$.

                    \begin{eqnarray*}
                        A(\vec{u}+\vec{v}) &=& A\begin{bmatrix}
                            u_1+v_1\\
                            \vdots\\
                            u_n+v_n
                        \end{bmatrix} \pause = \vec{a}_1(u_1+v_1) + \cdots + \vec{a}_n(u_n+v_n) \\
                        \pause &=& \vec{a}_1u_1 + \vec{a}_1v_1 + \cdots + \vec{a}_nu_n + \vec{a}_nv_n\pause = \vec{a}_1u_1 + \cdots + \vec{a}_nu_n + \vec{a}_1v_1 + \cdots + \vec{a}_nv_n \\
                        \pause &=& A\vec{u} + A\vec{v}
                    \end{eqnarray*}
                \end{proof}
        \end{frame}
    \begin{frame}{Revisiting Span}
        \begin{example}
            \small
            Determine if $\vec{b}=\begin{bmatrix} 6\\-2\\-10\end{bmatrix}$ is in 
                $
                    \Span{\begin{bmatrix}1\\0\\-2\end{bmatrix},\begin{bmatrix}0\\1\\-1\end{bmatrix}}
                $
        \end{example}
        \pause
        We solve either
        \begin{columns}
            \column{.5\textwidth}
            \begin{eqnarray*}
                1x_1 + 0x_2 &=& 6\\
                0x_1 + 1x_2 &=&-2\\
                -2x_1 + -1x_2 &=& -10
            \end{eqnarray*}
            \column{.5\textwidth}
            \[
                \aMat{cc|c}{
                    1&0&6\\
                    0&1&-2\\
                    -2&-1&-10
                }
            \]
        \end{columns}
        \pause
        Or solve $A\vec{x} = \vec{b}$ where
        \[
            A = \begin{bmatrix}
                1&0\\0&1\\-2&-1
            \end{bmatrix}\text{ and }\vec{b} = \begin{bmatrix}
                6\\-2\\-10
            \end{bmatrix}
        \]
    \end{frame}
    \begin{frame}{Span with Matrix-Vector Products}
        \begin{tcolorbox}
            If $A\in\R^{m\times n}$ has columns $\vec{a}_1,\dots,\vec{a}_n$ and $b\in\R^m$, then
            the matrix equation $A\vec{x}=\vec{b}$ has the same solution as 
            \[
                x_1\vec{a}_1 + \dots + x_n\vec{a}_n
            \]
        \end{tcolorbox}
        \vspace{70pt}
        \begin{theorem}
            The equation $A\vec{x}=\vec{b}$ has a solution if and only if $\vec{b}$ is a linear 
            combination of the columns of matrix $A$.
        \end{theorem}
    \end{frame}
    \begin{frame}{Span with General Vectors}
        \scriptsize
        We've answered questions about if a particular vector is in the span of others, but what
        about \rText{all} vectors?
        \pause
        \begin{example}
            Are all vectors $\vec{b}=\begin{bmatrix}b_1\\b_2\\b_3\end{bmatrix}\in\R^3$ in
            $\Span{
                \begin{bmatrix}
                    1\\1\\-3
                \end{bmatrix}, \begin{bmatrix}
                    1\\0\\2
                \end{bmatrix}, \begin{bmatrix}
                    6\\8\\6
                \end{bmatrix}
            }$?
        \end{example}
        \pause
        \begin{solution}
            Set up the augmented system and reduce!
            \[
                \aMat{ccc|c}{
                    1&1&6&b_1\\
                    1&0&8&b_2\\
                    -3&2&6&b_3
                }\pause\rightarrow\aMat{ccc|c}{
                    1&1&6&b_1\\
                    0&-1&2&b_2-b_1\\
                    -3&2&6&b_3
                }\pause\rightarrow\aMat{ccc|c}{
                    1&1&6&b_1\\
                    0&-1&2&b_2-b_1\\
                    0&5&24&b_3+3b_1
                }
            \]
            \[
                \pause\rightarrow\aMat{ccc|c}{
                    1&1&6&b_1\\
                    0&-1&2&b_2-b_1\\
                    0&0&34&b_3+3b_1+5(b_2-b_1)
                }.\pause\textnormal{ See that regardless of $b_1,b_2,b_3$, we can get an answer!}
            \]
        \end{solution}
    \end{frame}
    \begin{frame}
        \begin{theorem}
            Let $A\in\R^{m\times n}$. The following 4 statements are equivalent.
            \begin{enumerate}
                \item The equation $A\vec{x}=\vec{b}$ has a solution for every $\vec{b}\in\R^m$
                \item All vectors $\vec{b}\in\R^m$ can be written as linear combinations of columns
                    of $A$.
                \item The columns of $A$ span all of $\R^m$.
                \item $A$ has a pivot in every row.
            \end{enumerate}
        \end{theorem}
    \end{frame}
    \begin{frame}{Span Practice}
        For each of the following matrices, determine if their columns span all of $\R^3$.
        \begin{columns}
            \column{.5\textwidth}
            \begin{enumerate}
                \item \[
                        A = \begin{bmatrix}
                            3 & -2 & 7\\
                            1 & 1 & 4\\
                            6 & 1 & 19
                        \end{bmatrix}
                    \]
                    \iftoggle{showSolutions}{
                        \pause
                        \[
                            \begin{bmatrix}
                            3 & -2 & 7\\
                            1 & 1 & 4\\
                            6 & 1 & 19
                            \end{bmatrix}\pause\rightarrow\begin{bmatrix}
                            1 & 1 & 4\\
                            3 & -2 & 7\\
                            6 & 1 & 19
                            \end{bmatrix}
                        \]
                        \[
                            \pause\rightarrow\begin{bmatrix}
                            1 & 1 & 4\\
                            0 & -5 & -5\\
                            0 & -5 & -5
                            \end{bmatrix}\pause\rightarrow\begin{bmatrix}
                            1 & 1 & 4\\
                            0 & -5 & -5\\
                            0 & 0 & 0
                            \end{bmatrix}
                        \]
                        \pause\rText{NO!}
                    }{}
            \end{enumerate}
            \column{.5\textwidth}
            \begin{enumerate}
                    \addtocounter{enumi}{1}
                \onslide<1->\item \[
                        A = \begin{bmatrix}
                            3 & -2 & 1\\
                            1 & 1 & 2\\
                            3 & 4 & 1
                        \end{bmatrix}
                    \]
                    \iftoggle{showSolutions}{
                        \onslide<5->
                        \pause
                        \[
                            \begin{bmatrix}
                            3 & -2 & 1\\
                            1 & 1 & 2\\
                            3 & 4 & 1
                            \end{bmatrix}\pause\rightarrow\begin{bmatrix}
                            1 & 1 & 2\\
                            3 & -2 & 1\\
                            3 & 4 & 1
                            \end{bmatrix}
                        \]
                        \[
                            \pause\rightarrow\begin{bmatrix}
                            1 & 1 & 2\\
                            0 & -5 & -5\\
                            0 & -1 & -5
                            \end{bmatrix}\pause\rightarrow\begin{bmatrix}
                            1 & 1 & 2\\
                            0 & -5 & -5\\
                            0 & 0 & -4
                            \end{bmatrix}
                        \]
                        \pause\rText{YES!}
                    }{}
            \end{enumerate}
        \end{columns}
        \iftoggle{showSolutions}{}{
            \vspace{170pt}
        }
    \end{frame}
\end{document}
