\documentclass[xcolor=dvipsnames, aspectratio=169]{beamer}

\usepackage{3191Style}

\setbeamertemplate{itemize/enumerate subbody begin}{\scriptsize}

\date{Applications of Linear Systems}

\begin{document}


\begin{frame}{Modeling the US Economy}

  \begin{columns}

    \column{0.7\tw}
    
    \bi
  \ii In 1949, economist Wassily Leontief worked with $250,\!000$ data points produced by the United States BLS.
      \pause
  \ii Based on his researched, he initially divided the US economy into 500 different sectors.
      \pause
  \ii One of the world's highest powered computers, Harvard's Mark II, could not process so much information.
      \pause
  \ii The model had to be further classified into a system of 42 equations of 42 unknowns.
      \pause
  \ii Such models are now know as \alert{Leontief production models}.
      \pause
  \ii Mark II solved the system in 56 hours.
  \ei
  
  \column{0.3\tw}

  \onslide<3->
  \includegraphics[width=0.95\tw]{../Images/Chap1/fig-mach-ii.png}

  
  
  \end{columns}

  \vspace{20pt}
  \onslide<1->

  {\footnotesize  Based on \textit{Linear Algebra and Its Applications} by D. Lay, et al.}
  
  
  {\footnotesize  Image Credit: \alert{\href{http://lasierrainformatica.blogspot.com/2013/06/el-harvard-mark-ii.html}{http://lasierrainformatica.blogspot.com/2013/06/el-harvard-mark-ii.html}}}
  
\end{frame}

\begin{frame}{Input-Output Model}

  \bi
  \ii Instead of the 500 sectors Leontief identified, suppose the US economy is divided into 3 different sectors: Renewable Energy, Electricity, and Manufacturing.
  \pause
  \ii We can measure the total output for the year in each sector.
  \pause
 \ii We also know how each sector's output is divided among the other sectors.
  \pause
\ii We can summarize this economy with an \alert{input-output table}.
  \pause
  \ei

  \begin{center}
  \begin{tabular}{|l|c||c|c|c|}
    \hline
    &  & \multicolumn{3}{|c|}{Amount Purchased by } \\
    \hline
    Sector & Total Output (in Billions) & Renewables & Electricity & Manufacturing \\
    \hline
    Renewable Energy & $40$ & $10$ ($25$\%) & $25$ ($62.5$\%) & $5$ ($12.5$\%) \\ 
    \hline
   Electricity & $100$ & $7$ ($7$\%) & $18$ ($18$\%) & $75$ ($75$\%) \\
    \hline
    Manufacturing & $125$ & $20$ ($16$\%) & $50$ ($40$\%) & $55$ ($44$\%)\\
    \hline
 \end{tabular} 
  \end{center}
  
  \end{frame}


\begin{frame}{Exchange Tables}

  We can convert the input-output table to an \alert{exchange table} by giving the proportion of the output of each sector that is consumed by each of the other sectors.

  \pause
  \begin{center}
\begin{tabular}{|c|c|c|l|}
    \hline
    \multicolumn{3}{|c|}{Proportion of Output from:} &  \\
    \hline
   Renewable Energy & Electricity & Manufacturing & Purchased by \\
    \hline
    $0.25$ & $0.07$ & $0.16$ & Renewable Energy\\
    \hline
    $0.625$ & $0.18$ & $0.4$ & Electricity\\
    \hline
    $0.125$ & $0.75$ & $0.44$ & Manufacturing\\ 
    \hline
\end{tabular}
\end{center}

  \pause
{\small
 \bi
  \ii The total dollar amount of each sectors output is called the \alert{price} of that sector.
  \pause
  \ii Thus, \alert{$p_r$}, \alert{$p_e$}, and \alert{$p_m$} denote total outputs of Renewable Energy, Electricity, and Manufacturing sectors, respectively.
  \pause
  \ii Leontief proved that there exists equilibrium prices that can be assigned to the total outputs for each sector in such a way that the income of each sector exactly balances its expenses
  \pause
  \ii \colorb{How can we determine the equilibrium prices for each sector?}
  \ei
}

\end{frame}
  
\begin{frame}{Setting Up a System of Linear Equations}

  {\small
 \begin{center}
\begin{tabular}{|c|c|c|l|}
    \hline
    \multicolumn{3}{|c|}{Proportion of Output from:} &  \\
    \hline
   Renewable Energy & Electricity & Manufacturing & Purchased by \\
    \hline
    \colorg{$0.25$} & \colorb{$0.07$} & \alert{$0.16$} & Renewable Energy\\
    \hline
    $0.625$ & $0.18$ & $0.4$ & Electricity\\
    \hline
    $0.125$ & $0.75$ & $0.44$ & Manufacturing\\ 
    \hline
\end{tabular}
\end{center}
  }
  
 \bi
 \ii Down each column is the proportion of that sector which is purchased by each of the other three sectors.
 \ii Across each row we see for a given sector, what proportion of their inputs came from each sector's output.
 \ii For the renewable energy sector, we therefore have the following linear equation:
 \begin{align*}
   \mbox{output from renewables} &= (\mbox{input from renewables}) + (\mbox{input from electricity}) +  (\mbox{input from manufacturing})\\
   \colorg{p_r} &= \colorg{0.25 p_r} + \colorb{0.07 p_e} + \alert{0.16 p_m}\\
   \colorg{0.75 p_r}& - \colorb{0.07 p_e} - \alert{0.16 p_m} = 0
 \end{align*}
 \ei
 
\end{frame}

\begin{frame}{Setting Up a System of Linear Equations}

  {\small
 \begin{center}
\begin{tabular}{|c|c|c|l|}
    \hline
    \multicolumn{3}{|c|}{Proportion of Output from:} &  \\
    \hline
   Renewable Energy & Electricity & Manufacturing & Purchased by \\
    \hline
    \colorg{$0.25$} & \colorb{$0.07$} & \alert{$0.16$} & Renewable Energy\\
    \hline
    $0.625$ & $0.18$ & $0.4$ & Electricity\\
    \hline
    $0.125$ & $0.75$ & $0.44$ & Manufacturing\\ 
    \hline
\end{tabular}
\end{center}
  }


  Similarly deriving linear equations for the other sectors, we have the following system:
\[ \begin{array}{ccccccc}
  \colorg{0.75 p_r} & - & \colorb{0.07 p_e} & - & \alert{0.16 p_m} &=& 0\\
  \\
  \colorg{-0.625 p_r} & + & \colorb{0.82 p_e} & - & \alert{0.4 p_m} &=& 0\\
  \\
\colorg{-0.125 p_r} & - & \colorb{-0.75 p_e} & + & \alert{0.56 p_m} &=& 0
  \end{array} \]

\end{frame}

\begin{frame}{Finding the Equilibrium for the Economy}

  We have the following augmented matrix associated to the system:

    \[
        \aMat{ccc|c}{\colorg{0.75} & \colorb{-0.07} & \alert{-0.16} & 0 \\
  \colorg{-0.625} & \colorb{0.82} & \alert{-0.4} & 0 \\
  \colorg{-0.125} & \colorb{-0.75} & \alert{0.56} & 0
        }\rightarrow\aMat{ccc|c}{\colorg{1} & 0 & \alert{-0.28} & 0\\
  0 & \colorb{1} & \alert{-0.70} & 0 \\
  0 & 0 & \alert{0} & 0
        }
    \]
  Notice \alert{$p_m$} is a free variable, and we have equilibrium solution:

  \[ \mathbf{p} = \bbm p_r \\ p_e \\ p_m \ebm =  \bbm 0.28 p_m \\ 0.7 p_m \\ p_m \ebm = p_m \bbm 0.28 \\ 0.7 \\ 1 \ebm \]

  \bi
  \ii If this economy has \alert{$p_m = 125$ billion dollars},
  \ii Then if we want to ensure the economy is functioning at is equilibrium level (everything produced is used by other sectors):
  \bi
  \ii {\normalsize Set \colorg{$p_r = (0.28)(125) = 35$ billion dollars}, and}
  \ii  {\normalsize Set \colorb{$p_e = (0.7)(125) = 87.5$ billion dollars}.}
  \ei
  \ei
  
\end{frame}

\begin{frame}{Network Flow}

  \begin{columns}
    \column{0.65\tw}
    {\small
  \bi
  \ii A \alert{network} consists of a set of points, called \alert{nodes} with lines, called \alert{branches} connecting some or all of the nodes.
  \pause
  \ii The direction of the flow is indicated by each branch (are things flowing in or out of the node?).
  \pause
  \ii The flow amount (or rate) is either given or denoted by a variable.
  \pause
  \ii We assume the total flow into a network equals the total flow out of the network.
  \pause
  \ii The goal is to determine the flow in each branch when partial information is known.
  \pause
  \ii Network flows have applications to current flow through a circuit, flow of goods through supply chains, social networks, and \alert{urban planning} to name a few.
  \ei
    }
    
  \column{0.35\tw}

  \onslide<1->
  \includegraphics[width=0.95\tw]{../Images/Chap1/fig-social-network.png}
  \onslide<2->
  \includegraphics[width=0.95\tw]{../Images/Chap1/fig-node.png}
  
  \end{columns}

  \end{frame}

\begin{frame}{Traffic Flow in Baltimore}

  The network in the figure shows the flow of traffic (in vehicles per hour) over several one way streets in downtown Baltimore during a typical early afternoon. Determine the general flow pattern for the network.

  \begin{columns}
    \column{0.4\tw}
    
  \includegraphics[width=0.8\tw]{../Images/Chap1/fig-traffic-network.png}

  \pause
  \column{0.6\tw}
  
  \begin{tabular}{crcl}
    \hline
    Intersection & Flow in & & Flow out\\
    \hline
    A & $300+500$ & $=$ & $x_1+x_2$\\
    B & $x_2 + x_4$ & $=$ & $300+x_3$\\
    C & $100+400$ & $=$ & $x_4+x_5$\\
    D & $x_1+x_5$ & $=$ & $600$\\
    \hline
  \end{tabular}

  \pause
  \[
  \begin{array}{ccccccccccc}
    x_1 & + & x_2 &   &         &    &        &   &          &= & 800\\
           &     & x_2 & - & x_3 & + & x_4 &   &         &=& 300\\
           &     &        &    &        &    & x_4 & + & x_5 &=& 500\\
    x_1 &     &        &    &       &     &        & + & x_5 &=& 600\\
           &      &        &   & x_3 &    &        &     &        &=& 400
  \end{array}
  \]

  \end{columns}
   
  \end{frame}

\begin{frame}{Solving the System}

  We need to solve the following nonhomogeneous linear system of equations:

  \begin{columns}[T]
    \column{0.3\tw}

    \includegraphics[width=0.8\tw]{../Images/Chap1/fig-traffic-network.png}
    
      \column{0.7\tw}
\[   \begin{array}{ccccccccccc}
    x_1 & + & x_2 &   &         &    &        &   &          &= & 800\\
           &     & x_2 & - & x_3 & + & x_4 &   &         &=& 300\\
           &     &        &    &        &    & x_4 & + & x_5 &=& 500\\
    x_1 &     &        &    &       &     &        & + & x_5 &=& 600\\
           &      &        &   & x_3 &    &        &     &        &=& 400
\end{array} \]

\pause
\end{columns}

   We have an augmented matrix

    \[
        \aMat{ccccc|c}{
   $1$ & $1$ & $0$ & $0$ & $0$ & $800$\\
   $0$ & $1$ & $-1$ & $1$ & $0$ & $300$\\
   $0$ & $0$ & $0$ & $1$ & $1$ & $500$\\
   $1$ & $0$ & $0$ & $0$ & $1$ & $600$\\
   $0$ & $0$ & $1$ & $0$ & $0$ & $400$
        }\pause\rightarrow\aMat{ccccc|c}{
   $1$ & $0$ & $0$ & $0$ & $1$ & $600$\\
   $0$ & $1$ & $0$ & $0$ & $-1$ & $200$\\
   $0$ & $0$ & $1$ & $0$ & $0$ & $400$\\
   $0$ & $0$ & $0$ & $1$ & $1$ & $500$\\
   $0$ & $0$ & $0$ & $0$ & $0$ & $0$
        }
   \pause\rightarrow
   \left\{ \begin{array}{l}
     x_1=600-x_5\\
     x_2 = 200 + x_5\\
     x_3 = 400\\
     x_4 = 500-x_3\\
     x_5 \mbox{ is free}
     \end{array} \right.
    \]

\end{frame}

  \end{document}
