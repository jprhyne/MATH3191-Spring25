\documentclass[12pt]{exam}

\usepackage{mystyle}

\title{MATH 3191 Exam 2 Review}

\date{}

\begin{document}
\maketitle
\section{Format of Exam}
The exam will take the following format
\begin{enumerate}
    \item Some short response questions that require a sentence or two explaining why a statement
        is true or false (A beginning of what a proof might look like) $\sim$ 30 mins 
    \item Longer response questions that will require you to compute a numerical answer and show work
        on how it was achieved $\sim$ 30 mins
    \item Final question on lecture based content $\sim$ 15 mins
\end{enumerate}
\section{Written Homework Problems}
You should be able to do all of the Written Homework problems from HW7 up to and including HW10. If you want
help with your solutions or how to approach these problems, just let me know!
\section{MyOpenMath Problems}
I would pay attention to the following topics associated with each problem below
\begin{enumerate}
    \item HW7: 1, 2, 4, 9, 10, 13, 19
        \begin{enumerate}
            \item Be able to describe requirements for a vector space and demonstrate its properties
            \item Determine if a given set is a subspace of a known vector space ($\R^n,\mathcal{P}(\R)$)
            \item Describe what a null space/kernel is
            \item Describe what a column space is
            \item Describe what a spanning list is
        \end{enumerate}
    \item HW8: 1, 2, 3, 6, 7, 8, 10, 13, 18
        \begin{enumerate}
            \item Know the requirements for a basis of a vector space
            \item Determine a basis for a null space and column space
            \item Determine the $\mathcal{B}$-coordinates of a given vector for a particular basis
            \item Given the $\mathcal{B}$-coordinates of a vector in a particular basis, determine the vector in the standard basis
            \item Know how the dimension of the null space and column space relate.
        \end{enumerate}
    \item HW9: 1, 3, 4, 6, 7, 10, 11, 14, 16, 18
        \begin{enumerate}
            \item Determine the change of basis matrix for two given bases
            \item Convert a coordinate vector from one basis to another
            \item Determine if a given vector is an eigenvector
            \item Determine the eigenvalues of a matrix
            \item Determine the eigenvalue of a matrix power
        \end{enumerate}
    \item HW10: 1, 2, 5, 6, 9, 11, 15. (Pending lecture on 4-03 also look at 18, 21)
        \begin{enumerate}
            \item Diagonalize a matrix with real eigenvalues
            \item Compute a large power of a matrix ($A^n$ for $n\geq 4$)
            \item Know the requirements for a matrix to be diagonalizable and properties of diagonal matrices
            \item Compute complex eigenvalues of a $2\times 2$ or $3\times 3$ matrix
            \item These last objectives are only if we get to them on 4-03 in class
                \begin{enumerate}
                    \item Determine if a given matrix is a transition matrix
                    \item Compute the long run probabilities of Markov Chain
                \end{enumerate}
        \end{enumerate}
\end{enumerate}
\section{Lecture Based Problems}
\begin{enumerate}
    \item Explain what a similarity transformation is and what it does. IE: given $A = CBC^{-1}$, describe
        what each step of $CBC^{-1}\vec{x}$ is doing.
    \item Explain why using a different basis can make some problems easier (Like computing $A^n$).
    \item Explain what information diagonalization gives us about a matrix
    \item Be able to read Python Code from Lab 3 and MarkovChains.ipynb and explain at a high level
        what it is doing or what we expect output to be
    \item Pending lecture on 4-03 also be able to explain what a Markov chain is intuitively and algebraically
\end{enumerate}
\end{document}
