\documentclass[12pt]{exam}

\usepackage{mystyle}

\title{MATH 3191 Exam 1 Review}

\date{}

\begin{document}
\maketitle
\section{Format of Exam}
The exam will take the following format
\begin{enumerate}
    \item Some short response questions that require a sentence or two explaining why a statement
        is true or false (A beginning of what a proof might look like) $\sim$ 30 mins 
    \item Longer response questions that will require you to compute a numerical answer and show work
        on how it was achieved $\sim$ 30 mins
    \item Final question on lecture based content $\sim$ 15 mins
\end{enumerate}
\section{Written Homework Problems}
You should be able to do all of the Written Homework problems up to and including HW5. If you want
help with your solutions or how to approach these problems, just let me know!
\section{MyOpenMath Problems}
I would pay attention to the following topics associated with each problem below
\begin{enumerate}
    \item HW1: 5, 7, 10, 20, 21, 22
    \begin{enumerate}
        \item Be able to do row operations and identify how to get from one to another
        \item Solve a 2 by 2, 3 by 3, and 4 by 4 linear system.
        \item Identify the number of solutions to a linear system
        \item Add vectors
        \item Write a vector as a linear combination of others
        \item Know was a span is and how to compute it
    \end{enumerate}
    \item HW2: 1, 2, 8, 10, 14, 
    \begin{enumerate}
        \item Be able to multiply matrices and vectors
        \item Be able to convert between systems of equations and matrix equations
        \item Give a solution set to a system of equations
        \item Give solutions to a homogeneous linear system.
    \end{enumerate}
    \item HW3: 2, 3, 4, 7, 9, 15, 17, 18, 19
    \begin{enumerate}
        \item Determine conditions for a given matrix to have linearly dependent or independent
            columns
        \item Know properties of a linear transformation and determine if a function
            is linear.
        \item Be able to construct a matrix associated with a linear transformation
            given the image of some vectors
    \end{enumerate}
    \item HW4: 1, 2, 4, 8, 16, 17, 19, 22, 23
    \begin{enumerate}
        \item Identify what a given transformation is (translation, rotation, scaling, etc)
        \item Be able to construct the matrix associated with a given transformation 
            (translation, rotation, scaling, etc)
        \item Determine if a given transformation is injective/surjective
        \item Determine when matrix-multiplication is defined and compute the matrix product
        \item Compute the transpose of a matrix
        \item Determine how many solutions invertible and non-invertible matrices can have.
    \end{enumerate}
    \item HW5: 1, 10, 17, 20, 21, 22
    \begin{enumerate}
        \item Determine if a matrix is invertible using the invertible matrix theorem
        \item Identify what a given transformation is (translation, rotation, scaling, etc)
        \item Be able to construct the matrix associated with a given transformation 
            (translation, rotation, scaling, etc)
        \item Compute the determinant of a 2 by 2 matrix or larger triangular matrix
        \item Determine if a matrix is invertible based on the determinant
    \end{enumerate}
\end{enumerate}
\section{Lecture Based Problems}
\begin{enumerate}
    \item Explain what a span is and demonstrate a vector is in a span of other vectors
    \item Explain the difference between linear dependence and independence
    \item Explain what it means for a linear transformation to be Injective (one-to-one) and 
        Surjective (onto).
    \item Explain the properties of a determinant and methods to compute it.
    \item Be able to read Python Code from Labs 1 and 2 and describe what it does.
\end{enumerate}
\end{document}
